与现有研究工作相比,本项目的特色与创新之处体现在以下几方面:

\textbf{Template: 针对xxx问题,研究xxx方法、模型、算法,提升xxx准确性、可靠性。}

\begin{itemize}
    \item[(1)] \textbf{针对电子医疗记录的队列识别问题,本项目提出基于表现型的电
    子医疗记录表示学习,并提出新的弱监督学习方法,在仅有少量标注数据的情况下实现
    准确的队列识别}。目前针对电子医疗记录的表示学习局限于对医疗特征、一次就诊和
    某位患者的表示学习,缺乏针对表现型的表示学习,而患者的状态通常由多个表现型组
    合而成;另一方面,将患者的表示分解为多个表现型向量,有利于利用未标注数据,提
    高表现型表示学习的质量,因此,在大量电子医疗记录难以标注的情况下,本项目利用
    表现型字典表示学习进行队列识别具有重要的实际意义,是本项目一大特色和创新之
    处。
    \item[(2)] \textbf{针对电子医疗记录不规则性的问题,本项目通过融合医疗特征记
    录过程引入的医学偏差,提出电子医疗记录插补方法,可作为电子医疗记录分析的通用
    数据预处理方法}。为提高电子医疗记录的分析效率和降低模型设计的复杂度,本项目
    提出电子医疗记录实现缺失值插补模型,考虑电子医疗记录中存在医学偏差,且这种偏
    差与医疗特征记录的时间密切相关,本项目利用缺失标记矩阵自动学习各医疗特征的缺
    失规律,解决医学偏差为插补模型带来的影响,可有效提高电子医疗记录的插补质量,
    进而提高预测模型的准确性,是本项目相比于现有工作的重要创新之处。
    \item[(3)] \textbf{现有电子医疗记录的分析模型可解释性不足,本项目提出针对特
    征重要性和时间关联性两方面进行解释,通过引入混合注意力机制,解耦类RNN模型的
    隐含层,实现针对电子医疗记录的可解释分析模型}。可解释性对医疗领域的应用落地
    十分重要,也是在医疗机构应用机器学习模型的重要需求之一,现有电子医疗记录可解
    释性的研究较少,少量工作曾探索如何挖掘与预测结果相关性较大的记录,但缺少对模
    型行为的解释,即不能得到全局的特征重要性。本项目创造性地在分析记录的时间关联
    性的同时,总结特征的重要性,能更全面的展现模型的行为和决策理由,为在临床实践
    中应用奠定了重要的基础,也是本项目的重要特色与创新之处。
\end{itemize}