\subsection{年度研究计划}
本项目实施3年,具体年限为2023年1月1日至2025年12月31日。具体计划如下:

\textbf{第一阶段(2023年1月--2023年12月)},重点研究基于决策行为模拟的深度神经网
络黑盒测试方法,实现利用副本模型进行知识萃取的技术,并提出适用于复杂神经网络的测
试覆盖指标,具体包括:
\begin{itemize}[itemindent=2em]
    \item[(1)] 研究基于知识萃取的决策行为模拟方法;
    \item[(2)] 提出适用于复杂神经网络的具有可解释性的测试覆盖指标; 
    \item[(3)] 申请发明专利1项;
    \item[(4)] 在国际期刊或会议上发表研究论文1-2篇,参加国际会议1-2次,并在会议上
    做论文成果报告。
\end{itemize}

\textbf{第二阶段(2024年1月--2024年12月)},重点研究基于层次语义理解的深度神经网
络白盒测试技术,实现对原复杂模型的决策路径抽取和副本模型训练技术,并提出新的结构
覆盖指标,具体包括:
\begin{itemize}[itemindent=2em]
    \item[(1)] 研究对复杂模型的决策路径抽取和副本模型构建方法;
    \item[(2)] 提出适用于复杂神经网络的白盒测试覆盖指标; 
    \item[(3)] 申请发明专利1项;
    \item[(4)] 在国际期刊或会议上发表研究论文1-2篇,参加国际会议1-2次,并在会议
    上做论文成果报告。
\end{itemize}

\textbf{第三阶段(2025年1月--2025年12月)},重点研究基于反馈偏置的自适应测试集生
成方法,构建测试反馈机制,实现自适应的测试集生成方法,具体包括:
\begin{itemize}[itemindent=2em]
    \item[(1)] 构建神经网络测试反馈机制,支持测试结果和行为特征的关联挖
    掘;
    \item[(2)] 基于白盒/黑盒测试反馈偏置,实现自适应的测试集生成方法;
    \item[(3)] 在国际期刊或会议上发表研究论文1-2篇,参加国际会议1-2次,并在会议上
    做论文成果报告; 
    \item[(4)] 项目总结,完成结项报告,准备验收。
\end{itemize}

\subsection{预期研究成果}

本项目的预期研究成果包括以下几个方面:
\begin{itemize}[itemindent=2em]
    \item[(1)] 在CCF-A类推荐期刊/会议或其他SCI上发表高水平论文3-6篇;;
    \item[(2)] 申请专利2项;
    \item[(3)] 培养研究生2-3人; 
    \item[(4)] 完成一个面向复杂神经网络的可解释测试框架;
    \item[(5)] 开源相关研究工作,供用户下载,并提供说明和使用文档。
\end{itemize}
