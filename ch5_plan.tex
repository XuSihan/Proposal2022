\subsection{年度研究计划}
本项目实施3年,具体年限为2023年1月1日至2025年12月31日。具体计划如下:

\textbf{第一阶段(2023年1月--2023年12月)},重点研究基于决策行为模拟的深度神经网
络黑盒测试方法,实现利用副本模型进行知识萃取的技术,并提出适用于复杂神经网络的测
试覆盖指标,具体包括:
\begin{itemize}[itemindent=2em]
    \item[(1)] 进行文献收集,整理,进而完善与细化研究方案。
    \item[(2)] 研究基于层次语义理解的测试覆盖目标抽取方法;
    \item[(3)] 提出适用于复杂神经网络的具有可解释性的测试覆盖指标;
    \item[(4)] 在国际期刊或会议上发表研究论文1-2篇,参加国际会议1-2次,并在会议上
          做论文成果报告,申请发明专利1项。
\end{itemize}

\textbf{第二阶段(2024年1月--2024年12月)},重点研究基于层次语义理解的深度神经网
络白盒测试技术,实现对原复杂模型的决策路径抽取和副本模型训练技术,并提出新的结构
覆盖指标,具体包括:
\begin{itemize}[itemindent=2em]
    \item[(1)] 研究对复杂模型的决策路径模拟方法;
    \item[(2)] 提出基于决策路径模拟的黑盒测试覆盖指标;
    \item[(3)] 本年度参加一次全国软件工程大会(NASAC 2024);
    \item[(4)] 在国际期刊或会议上发表研究论文1-2篇,参加国际会议1-2次,并在会议上做论文成果报告,申请发明专利1项。
\end{itemize}

\textbf{第三阶段(2025年1月--2025年12月)},重点研究基于反馈偏置的自适应测试集生
成方法,构建测试反馈机制,实现自适应的测试集生成方法,具体包括:
\begin{itemize}[itemindent=2em]
    \item[(1)] 研究利用基于实例的解释方法选取代表数据;
    \item[(2)] 研究融合反馈偏置的自适应方法选取边界数据;
    \item[(3)] 本年度邀请一次国内专家进行相关领域报告一次;
    \item[(4)] 在国际期刊或会议上发表研究论文1-2篇,参加国际会议1-2次,并在会议上
          做论文成果报告;
    \item[(5)] 项目总结,完成结项报告,准备验收。
\end{itemize}

\subsection{预期研究成果}

针对深度学习模型的可解释测试技术研究工作预期可以在构建可解释测试覆盖指标上取得以下研究成果:
\begin{itemize}[itemindent=2em]
    \item[(1)] 提出基于层次语义理解的测试覆盖目标抽取技术;
    \item[(2)] 构建具有层次语义逻辑的白盒测试覆盖指标;
    \item[(3)] 提出基于知识蒸馏的决策路径模拟技术;
    \item[(4)] 构建体现决策逻辑和决策依据的黑盒测试覆盖指标;
\end{itemize}

在可解释测试集生成上取得以下研究成果:
\begin{itemize}[itemindent=2em]
    \item[(1)] 提出基于可解释测试覆盖指标的测试数据生成技术;
    \item[(2)] 提出利用基于实例的解释方法选取代表数据的技术;
    \item[(3)] 构建反馈偏置机制,提出适用于边界测试数据选取的自适应测试方法;
    \item[(4)] 对提出的解释性方案效果通过基准数据集进行真实验证,并完成与相关工作的比较;
    \item[(5)] 构建完成面向基于人工智能的自动驾驶软件在转向/转角预测任务上的测试数据集、测试报告库。
\end{itemize}

本项目在完成深度学习模型的可解释测试技术研究的基础上,预计发表高水平学术论文3-6篇,其中在CCF-A类推荐期刊/会议上发表学术
论文3篇;完成一个面向深度学习模型的可解释测试框架,并在自动驾驶任务上进行真实验证;开源相关研究工作,并提供说明和使用文
档;申请专利2项;培养研究生2-3人。
