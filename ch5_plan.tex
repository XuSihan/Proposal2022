\subsection{年度研究计划}
本项目实施3年,具体年限为2021年1月1日至2023年12月31日。具体计划如下:

\textbf{第一阶段(2021年1月--2021年12月)},重点研究基于表现型的自动队列识别,向
医生请教DKD慢性肾病和感染性休克的表现型,并请专家进行数据标注,具体包括:
\begin{itemize}[itemindent=2em]
    \item[(1)] 研究整理DKD慢性肾病和感染性休克的常见的表现型并标注;
    \item[(2)] 研究弱监督条件下的自动队列识别方法; 
    \item[(3)] 实现针对DKD慢性肾病和感染性休克的队列识别;
    \item[(4)] 在国际期刊或会议上发表研究论文2-3篇,参加国际会议1次,并在会议上
    做论文成果报告。
\end{itemize}

\textbf{第二阶段(2022年1月--2022年12月)},重点研究融合医学偏差的电子医疗记录
查插补方法,并实现针对DKD慢性肾病和感染性休克的队列识别和数据补全,具体包括:
\begin{itemize}[itemindent=2em]
    \item[(1)] 研究融合医学偏差的电子医疗记录查插补方法;
    \item[(2)] 实现针对DKD慢性肾病和感染性休克的数据插补预处理;
    \item[(3)] 在国际期刊或会议上发表研究论文2-3篇,参加国际会议1-2次,并在会议
    上做论文成果报告; 
    \item[(4)] 申请发明专利1项。
\end{itemize}

\textbf{第三阶段(2023年1月--2023年12月)},重点针对电子医疗记录分析的可解释预测
模型,构建端到端的电子医疗记录分析解决方案,具体包括:
\begin{itemize}[itemindent=2em]
    \item[(1)] 构建电子医疗记录的可解释预测模型,支持特征重要性和时间关联度挖
    掘;
    \item[(2)] 针对DKD慢性肾病和感染性休克的构建可解释预测模型,研发可视化模型解
    释界面,并与医学专家讨论模型的可解释结果;
    \item[(3)] 在国际期刊或会议上发表研究论文2-3篇,参加国际会议1次,并在会议上
    做论文成果报告; 
    \item[(4)] 申请发明专利1项;
    \item[(5)] 项目总结,完成结项报告,准备验收。 
\end{itemize}

\subsection{预期研究成果}

本项目的预期研究成果包括以下几个方面:
\begin{itemize}[itemindent=2em]
    \item[(1)] 在国内外高水平期刊、会议上发表论文8篇以上,全部被SCI/EI检索,其中
    有重要影响的论文4篇以上;
    \item[(2)] 申请专利2项;
    \item[(3)] 培养研究生3-4人; 
    \item[(4)] 完成一个基于深度学习的端到端的电子医疗记录分析框架;
    \item[(5)] 开源相关研究工作,供用户下载,并提供说明和使用文档。
\end{itemize}
