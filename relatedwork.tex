\subsection{国内外相关工作}\label{relatedwork}

% 根据与本项目的相关性,本节从覆盖充分性指标、神经网络测试数据生成、深度学习框架测试以及第三方组件漏洞挖掘四个方面介绍和分析国内外研究现状。

根据与本项目的相关性,本节从可解释性研究、深度学习模型测试度量指标和测试集生成三个方面介绍和分析国内外相关工作。

\textbf{在安全攸关任务上,测试数据集生成方法DeepSuite\citess{xu2021deepsuite}对
基于深度神经网络的自动驾驶软件的测试进行了探索性尝试和研究(本项目申请人作为该文
章的第一作者,且文章已收录于IEEE T-ITS,中科院SCI一区期刊)}。该方法基于多目标搜
索,对测试集种子进行测试集和测试数据两个级别,在Udacity自动驾驶数据集上证明了所
生成测试集具有高错误检测能力。




%研究现状:
%1.覆盖充分性指标体系--缺点,找不到失效原因,不可解释
%2.测试数据集生成方法--本项目申请人--缺点,仅仅针对神经网络,也就是智能组件
%3.框架测试方法--缺点,忽略了其它第三方组件漏洞,可以攻击框架漏洞,给智能组件带来安全威胁,也可以直接激活智能组件的脆弱性。
%4.第三方组件漏洞挖掘--缺点,单组建非跨组件,更缺少组件对人工智能模型各阶段的影响分析。

\subsubsection{可解释性研究}
{人工智能的可解释性研究是最近几年人工智能领域和软件工程领域(可信人工智能)的研
究热点}。浙江大学的纪守领老师团队对于目前机器学习和深度学习模型的可解释性研究进
行了详细的调研和总结,调研表明\textbf{目前人工智能的可解释性研究仍旧处于初级阶
段,并且学术界对深度学习模型的可解释性仍旧缺乏统一的认识}\citess{jishouling}。
模型可解释性总体上可分为\textbf{事前(ante-hoc)可解释性}和\textbf{事后
(post-hoc)可解释性}。其中,事前可解释性指通过训练结构简单、可解释性强的模型或将
可解释性结合到模型结构中,使模型本身具备可解释能力。事后可解释是指通过可解释性方
法来对已经训练好的模型进行解释\citess{jishouling}。

事前可解释性无需额外的信息就可以理解模型的决策过程或决策依据,通常采用结构简单、
易于理解的模型来实现,例如线性回归、决策树、朴素贝叶斯等。事前可解释性要求模型的
决策过程或决策依据可模拟,模型的每一个输入以及每一维特征都有直观的解释。对于朴素
贝叶斯模型的预测,可以很容易地转化为单个特征值的贡献;对于线性模型,其权重则直接
反映了特征重要性;对于决策树模型,每一条决策路径所对应的条件规则都为最终的分类结
果提供了解释。为了提高简单线性模型的准确率,同时保留其事前可解释性,广义加性模型
通过简单的线性函数组合每个单特征模型得到最终决策。另外,事前可解释性也可通过引入
注意力机制来达到,注意力权重矩阵直接体现了模型在决策过程中感兴趣的区域,从而对模
型进行一定程度的解释。

事后可解释性发生在模型训练之后,通过构建解释方法对模型的工作机制、决策行为和决策
依据进行解释。早期针对事后可解释性的研究主要集中于通过规则提取得到对复杂模型整体
决策逻辑的理解。然而,规则提取只能提供近似解释,不一定能反映待解释模型的真实行
为,且受规则复杂度的制约。由于整体解释模型行为较为困难,很多研究以输入样本为导
向,通过分析输入样本的每一维特征对模型最终决策结果的贡献来实现事后可解释性,典型
的方法包括敏感性分析解释、局部近似解释、梯度反向传播解释、特征反演解释等。然而,
现有解释方法很难保证解释结果的准确性和一致性,为此,Chu等人提出了一种解释方法,
可为分段线性神经网络家族模型提供精确一致的解释。然而,该方法只能解释线性神经网络
模型,无法应用于解释非线性神经网络模型。

\textbf{尽管已经有很多针对机器学习和深度学习模型本身的可解释性研究,但是对于本项
目关注的针对深度学习系统测试方面的可解释性研究工作还十分有限。}Kim等人
\citess{Kim2019Guiding}提出一种基于“意外度”的测试覆盖指标,通过度量测试数据与训
练集的距离/相似度,对测试覆盖指标进行一定的可解释性分析。Xie等人
\citess{Xie2021NPC}提出了神经元路径覆盖指标,基于神经网络中决策图的控制流和数据
流,提出了两种路径覆盖的变体来衡量测试集的充分性。虽然这些方法在一定程度上解释了
针对深度学习模型的测试覆盖指标,但他们局限于模型推理时神经元的取值,缺乏语义层面
上对模型决策行为的解释,难以根据测试结果检查模型是否在以符合人类认知的形式正常工
作。\textbf{因此,可解释性方法的能力在深度学习模型测试领域并没有得到准确的说明和
研究}。

\iffalse
\textbf{尽管已经有很多针对机器学习和深度学习模型本身的可解释性研究,但是对于本项
目关注的针对深度学习系统测试方面的可解释性研究工作还十分有限。}Kim等人
\citess{Kim2019Guiding}提出一种基于“意外度”的测试覆盖指标,通过度量测试数据与训
练集的不同距离,评估测试集对样本输入空间的覆盖度,从而对测试目标进行一定的可解释
性分析。Xie等人\citess{Xie2021NPC}提出了神经元路径覆盖指标,类似于传统的程序控制
流图,首先从深度神经网络中提取决策图用来表示模型的决策逻辑,然后基于决策图的控制
流和数据流,该方法提出了两种路径覆盖的变体来衡量测试数据在执行决策逻辑时的充分
性。该测试方法在一定程度上反映出模型的决策逻辑,但由于模型本身缺乏可解释性,难以
从控制流或数据流路径上辅助开发人员找到模型失效的原因,从而帮助修复模型。但是该方
法依赖于模型推理过程中的神经元取值,缺乏对模型决策行为的解释,难以检查模型是否在
以符合人类认知的形式正常工作。
\fi

\subsubsection{测试度量指标}
\textbf{测试覆盖指标是衡量测试集对深度学习模型测试充分性的标尺,是深度学习测试的重要研究
问题之一。}对于传统软件,若测试集遍历了待测软件所有的语句、分支和路径,则在一定程
度上表明测试集对软件的功能进行了充分性测试\citess{hilton2018large}。对于深度神经
网络而言,其本身的高维连续特性导致测试集很难遍历所有可能的输入空间,为提高测试集
多样性,目前有很多研究提出了关于深度神经网络的结构覆盖指标,主要分为基于单个神经
元取值和基于多个神经元取值组合两类。

%,从不同角度测试衡量测试集对模型的覆盖充分性。


\iffalse
\cref{tab:coverage_criteria}总结了现有神经网络测试覆盖指标,其中$m$表示训练
集规模,$n$表示神经元个数,$l$表示神经网络层数。根据覆盖思想的不同,可分为以下六
类:

\begin{table}[htp]
	\renewcommand\arraystretch{1.5}
	\small
	\centering
	\caption{现有神经网络覆盖充分性指标}
	\label{tab:coverage_criteria}
	% \begin{tabular}{p{3cm}p{5cm}p{1cm}p{1cm}p{2cm}}
	\begin{tabular}{ccccc}
		\toprule
		\textbf{序号} & \textbf{主要思想} & \textbf{覆盖指标}       & \textbf{复杂度} & \textbf{文献号}                                                  \\
		\midrule
		1             & 基于单个神经元取值  & 神经元覆盖、$k$-多区间覆盖                    & $O(n)$          & \cite{ma2018deepgauge}\cite{Pei2019DeepXplore} \\
		2             & 训练集神经元边界  & 神经元边界覆盖、强激活覆盖                    & $O(nm)$          & \cite{ma2018deepgauge}                                           \\
		3             & 与训练数据分布的距离  & 意外覆盖,平均偏差等                & $O(nm)$         & \cite{Kim2019Guiding}\cite{Tian2019Testing}            \\
		4             & 神经元激活通路覆盖    & 符号-符号覆盖、距离-符号覆盖等 & $O(nl)$ & \cite{Wang2019DeepPath}\cite{Sun2018Testing} \\
		5             & 神经元的状态转换        & 状态级别覆盖、转换级别覆盖                         & $O(n)$          & \cite{Du2018DeepCruiser}                                         \\
		6             & 神经元组合测试    & $t$-way组合稀疏覆盖、密集覆盖等 & $O(n^2)$ & \cite{ma2019deepct} \\
		\bottomrule
	\end{tabular}
\end{table}

\fi



%其中涉及到的覆盖指标主要有神经元覆盖率(NC),$k$区间覆盖率(KMNC),重要性驱动覆盖
%(IDC),神经元边界覆盖率(NBC),强神经元覆盖率(SNC),重要神经元覆盖(INC),意外覆
%盖(SC),神经元激活向量距离(NAVD),平均偏差(MD),重要神经元通路覆盖率(INPC),强
%激活通路覆盖率(SAPC),符号-符号覆盖(SSC),距离-符号覆盖(DSC),符号-值覆盖
%(SVC),距离-值覆盖(DVC),状态级别覆盖(BSC),转换级别覆盖(BTC),$t-way$组合稀疏
%%%%覆盖($t-way$ CSC),$t-way$组合密集覆盖($t-way$
%CDC),$(p,t)$完整性覆盖($(p,t)$ C)等指标。

%DeepXplore~\cite{Pei2019DeepXplore}、DeepGauge~\citess{ma2018deepgauge}、IDC~\citess{Gerasimou2020Importance}
%等工作主要围绕神经元覆盖率、K区间覆盖率、重要性驱动覆盖等测试覆盖指标。


\textbf{基于单个神经元取值},受传统软件的结构覆盖测试启发,早期针对深度学习模型
的测试覆盖指标以神经元为基本单位,通过度量单个神经元取值被覆盖的程度来评估测试充
分性。Pei等人\citess{Pei2019DeepXplore}首次提出了针对深度学习模型白盒测试指标,
即神经元覆盖,将取值高于阈值的神经元视为被激活,并计算被激活神经元的比例。 Ma等
人\citess{ma2018deepgauge}提出了$k$-多区间覆盖指标,基于每个神经元在训练集上的取
值范围,将其划分为多个取值区间,并计算对神经元值区间的覆盖率。Ma等人
\citess{ma2018deepgauge}提出神经元边界覆盖指标和强激活覆盖指标,计算测试集中神经
元的值超过训练集的上下边界的比例。Gerasimou等人\citess{Gerasimou2020Importance}
设计了通过衡量神经元对分类结果的影响比例提出了重要性驱动覆盖标准。

\textbf{基于多个神经元取值组合},由于单个神经元对模型内部活动状态的描述存在局限
性,目前基于多个神经元取值组合的工作
\citess{Kim2019Guiding,Tian2019Testing,Wang2019DeepPath,Sun2018Testing,ma2019deepct,Xie2021NPC}
占有的比重更大。Kim等人\citess{Kim2019Guiding}提出了“意外”覆盖指标,衡量模型的测
试数据和训练集在推理时隐藏层向量表示之间的距离,并将这种距离划分为多个区间,通过
计算测试集对区间的覆盖比例度量测试集的覆盖充分性。Tian等人
\citess{Tian2019Testing}在图像分类任务上提出一种基于神经元激活频率的白盒测试框架
DeepInspect,用来检测分类器中的混淆和偏差错误。 {基于神经元激活通路},Wang 等人
\citess{Wang2019DeepPath}提出了一组针对神经网络模型的路径驱动的测试度量指标,能
够更好地识别对抗性样本。 Sun等人\citess{Sun2018Testing}提出将传统的MC/DC覆盖标准
应用于深度神经网络,并在此基础上采用梯度搜索的方法生成新的测试集。Ma等人
\citess{ma2019deepct}将组合测试应用于神经网络,提出了$t$-way组合稀疏覆
盖、$t$-way组合密集覆盖和$(p,t)$完整性覆盖等指标。Xie等人\citess{Xie2021NPC}基于
神经网络决策图的控制流和数据流,提出了两种路径覆盖的变体来衡量测试集的充分性。


\textbf{虽然现有测试覆盖指标在一定程度上反映了测试数据的多样性,但现有指标依赖于
对神经元取值的覆盖程度,缺少对高层次语义表示覆盖的研究,因此对测试充分性的度量缺
乏可解释性。}

%现有指标伸缩性较差,其计算时间和指标有效性难以应对实际大规模模型。本项目拟提出
%基于知识萃取的可解释测试覆盖指标,将知识蒸馏和知识回顾应用于神经网络模型测试,
%从而提升测试指标的伸缩性和可解释性。


%,导致在这些指标引导下的测试数据集生成缺乏符合人类认知的测试目的,难以帮助测试人员诊断和调试模型。

\subsubsection{测试数据集生成}

为提高深度学习模型的可靠性,使用足够的测试输入对其一般行为和各种边界条件下的行为
进行充分测试是十分必要的。在对深度学习模型的测试中,如何生成更具代表性的和更容易
暴露模型错误行为的测试数据已成为深度学习测试的一个研究重点。如
\cref{tab:testingDataGen}所示,现有测试数据生成方法可分为以下三类:

\begin{table}[htp]
	\renewcommand\arraystretch{1.5}
	\small
	\centering
	\caption{深度学习模型测试数据生成方法总结}
	\label{tab:testingDataGen}
	% \begin{tabular}{cp{5cm}p{2cm}cp{2cm}}
	\begin{tabular}{cccccc}
		\toprule
		\textbf{序号} & \textbf{算法思想} & \textbf{评价方法}               & \textbf{测试数据} & \textbf{文献号}             \\
		\midrule
		1             & 模糊测试          & 测试覆盖率、效率 & 图像、文本 & \cite{Odena2019TensorFuzz}\cite{Guo2018DLFuzz}\cite{xie2019coverage} \\
		2             & 符号执行          & 测试覆盖率、像素重要性等                              & 图像、代码              & \cite{Gopinath2018Symbolic}\cite{Sun2018Concolic} \\
		3             & 对抗样本          & 准确率、失真度、人类对比评价等 & 图像 & \cite{Xiao2018Spatially}\cite{Wicker2018FeatureGuided}\cite{He2018Decision} \\
		\bottomrule
	\end{tabular}
\end{table}


\begin{itemize}
	\item \textbf{基于模糊测试的思想},通过随机或者特定规则将种子输入进行变换,
	生成新的测试数据,观察模型在边界条件下是否会发生错误。Guo等人
	~\cite{Guo2018DLFuzz}首次提出神经网络模糊测试框架DLFuzz,用于指导生成暴露模
	型错误行为的测试数据。Odena等人~\cite{Odena2019TensorFuzz}利用测试覆盖指标指
	导模糊测试。在此基础上,Xie等人~\cite{xie2019coverage}提出了一个自动化模糊测
	试框架DeepHunter,使用6种测试覆盖指标实现了在深度学习模型开发和部署两个阶段
	的自动化测试。

	\item \textbf{基于符号执行},Sun等人~\cite{Sun2018Concolic}在测试覆盖指标的
	基础上提出了DeepConcolic,结合具体执行和符号分析,提高神经网络的测试覆盖
	率。Gopinath等人~\cite{Gopinath2018Symbolic}提出了一种轻量级符号执行技术并将
	其应用于图像分类算法的测试,以解决重要像素的识别以及创建1像素和2像素攻击等关
	键问题。

	\item \textbf{基于对抗样本的方法},通过向原始样本添加微小扰动的方式产生对抗
	      样本,使深度学习模型做出错误预测。在白盒攻击方面,Xiao等人
	      ~\cite{Xiao2018Spatially}提出了基于空间变换的图像对抗样本生成方法。He
	      等人~\cite{He2018Decision}提出了一种针对区域分类的对抗样本生成方法。在
	      黑盒攻击方面,Wicker等人~\cite{Wicker2018FeatureGuided}提出一种特征引
	      导的鲁棒性测试方法,通过双方博弈游戏的方式确定特征和操作像素点,并利用
	      蒙特卡罗树搜索算法逐步探索博弈状态空间来生成对抗性样本。
\end{itemize}


在深度学习模型部署前找到容易导致模型错误行为的输入数据是必要的。现有测试数据生成
方法主要分为两类:一类受传统软件测试方法启发,在输入种子数据的基础上允许在语义大
致不变的前提下对输入进行变异,以提高覆盖率为导向生成新数据;另一类基于对抗样本,
通过基于梯度等方法搜索导致模型错误预测的最小扰动,生成新测试数据。{\kaishu 然
而,由于神经网络的黑盒特性,这两类测试数据生成方法虽然能够暴露模型错误行为,但测
试结果缺乏可解释性,测试人员很难掌握测试成功或失效的原因,因此除扩充训练集外,对
模型修复的作用较少。本项目拟提出具有可解释性的深度学习模型测试框架,从模型和测试
数据两个角度提高深度学习测试的可解释性。}













\subsubsection{测试数据选择方法}

由于深度学习模型的输入数据的样本空间通常较大,而人工标注测试预言的成本较高,因此
很难在系统部署前检测每个潜在数据样本的预测正确性。为解决这个问题,部分研究提出测
试数据选择方法,从大规模无标注数据中选择出优先标注和测试的输入数
据。\cref{tab:testingDataPri}总结了现有测试数据选择方法,根据选择思想的不同可分
为以下三类:

\begin{table}[htp]
	\renewcommand\arraystretch{1.5}
	\small
	\centering
	\caption{深度学习模型测试数据集优选方法总结}
	\label{tab:testingDataPri}
	\begin{tabular}{cccc}
		\toprule
		\textbf{序号} & \textbf{算法思想}  & \textbf{测试对象} & \textbf{文献号}                                                    \\
		\midrule
		1             & 采用变异测试的思想       & 图像              & \cite{Wang2021Prioritizing}\cite{Ma2018DeepMutation}  \cite{Liu2022DeepState}                    \\
		2             & 基于测试数据的执行结果       & 图像、自然语言  & \cite{Byun2019Input}\cite{Shen2020MultipleBoundary}\cite{Feng2020DeepGini}\cite{Hu2022AnEmpirical}\cite{Gao2022Adaptive} \\
		\bottomrule
	\end{tabular}
\end{table}

\begin{itemize}

	\item \textbf{基于变异测试的思想},部分研究对神经网络模型进行变异,优选能够
	      检测出较多变异模型的测试数据进行标注。Ma等人
	      ~\citess{Ma2018DeepMutation}从训练数据、训练代码和深度学习模型三个角度
	      注入错误,通过错误检出率衡量测试数据质量。Wang等人
	      ~\citess{Wang2021Prioritizing}利用learning-to-rank算法
	      ~\citess{liu2019exploiting}构建测试数据排序模型,能够针对不同的深度学
	      习模型自动优选具有较高错误检测能力的测试数据。Liu等人
	      ~\citess{Liu2022DeepState}提出一种针对循环神经网络的测试集优选方法
	      DeepState,通过捕获神经元状态的变化来识别可能预测错误的测试数据。  

	\item \textbf{基于测试数据的执行结果}, Byun等人~\citess{Byun2019Input}从置
	      信度、不确定性和意外度等三个指标衡量测试数据的错误检测能力,并评估测试
	      数据对于模型再训练的提升。Feng等人~\citess{Feng2020DeepGini}提出一种基
	      于数据统计的测试集优选方法DeepGini,通过输出概率分布识别可能被错误分类
	      的测试数据。Shen等人~\citess{Shen2020MultipleBoundary}将测试数据聚类到
	      深度学习模型的多个边界区域中,从边界区域中均匀选择样本以确保每个边界都
	      有足够的测试数据。Hu等人~\citess{Hu2022AnEmpirical}提出了面向测试集优
	      选的数据分布敏感指标,用来减轻数据分布差异对标注数据集选择的影响。

\end{itemize}

{\kaishu 现有针对深度学习模型的测试数据集优选方法,主要集中于测试数据是否能够反
映模型错误行为,而对测试数据导致模型错误的原因缺乏研究。本项目拟提出基于反馈偏置
的自适应测试集生成,在可解释测试覆盖度量的基础上,选取代表性的样本来解释模型结
果,估算模型性能;同时,自适应找到导致模型错误行为的测试数据优先标注,提高测试集效率。}










\subsubsection{深度学习测试可解释性}




由于深度学习模型的黑盒特性,开发人员很难将测试结果与模型行为联系起来,深度学习测
试能够提供的有效修复信息较少。此外,Chen等人~\citess{Chen2020Practical}提出了一
种可解释的测试数据选择方法,利用基于实例的可解释算法MMD-critic选取具代表性的测试
输入来估算模型在整个测试集上的准确性。该方法的可解释性主要体现在测试数据的选择流
程上,而对测试反馈的可解释性仍然不足。

{\kaishu 虽然结构覆盖测试能够在一定程度上提高测试集的充分性,但由于模型本身的黑
盒特性,现有的深度学习模型测试工作也缺乏可解释性,开发人员很难建立测试结果与模型
行为之间的联系。本项目拟提出针对深度学习模型的可解释测试框架,采用知识蒸馏和知识
回顾的思想,从白盒和黑盒两个角度分别提出可解释测试覆盖指标,通过测试反馈辅助开发
人员修复模型。}

%然而在现实情形中,智能软件大量依赖基础库和第三方依赖库,其脆弱性来源包括智能组件脆弱性、非智能组件脆弱性以及跨组件脆弱性,而智能组件、基础框架组件以及其它第三方组件之间的交互影响尚不明确,第三方开源组件数量极大,版本较多,更新频繁,组件间依赖关系复杂,挖掘可能存在漏洞的脆弱组件难度较大。



\subsubsection{知识蒸馏技术}

知识蒸馏是指一种教师-学生(Teacher-Student)模型训练结构,其目标是以尽可能小的代
价将教师模型学到的知识迁移到简单的学生模型中~\citess{Gou2021KnowledgeDA}。该技术
被广泛应用于模型压缩,目前在图像、文本、音频等多种模态数据的处理任务中均有应用。
在计算机视觉方面,Hou等人~\citess{Hou2020CVPR}通过传输图像样本中不同区域之间的结
构关系,将教师网络学习到的场景结构知识迁移给学生执行道路标记分割任务。Fu等人
~\citess{Fu2020Ultrafast}提出将教师模型学习到的空间和时间知识迁移到低分辨率的轻
量级时空网络中来执行视频注意预测任务,高分辨率数据上训练得到的知识在低分辨率图片
处理任务上具有重要价值,同时能够降低对于计算机资源和存储的要求。在自然语言处理方
面,Wang等人~\citess{Wang2020StructureLevelKD}和Mukherjee等人
~\citess{Mukherjee2020XtremeDistilMD}将若干单语言的教师模型学习到的结构知识和内
部特征迁移到统一的多语言学生模型,来得到轻量级的多语言序列标注模型,而且其性能表
现比原有复杂模型更优。在语音识别领域,Aguilar等人
~\citess{Aguilar2020KnowledgeDF}提出将教师网络的多个Transformer层的特征知识压缩
到学生的单个Transformer层中,Liu等人~\citess{Liu2019EndtoEndST}提出引入自适应层
来压缩Transformer结构,在保持transformer对长序列学习问题的优越性的同时,也减轻了
它过大的参数规模,便于计算和存储。

部分研究将知识蒸馏用于生成具有可解释性的学生模型。Liu等人
~\citess{Liu2018ImprovingTI}将深度神经网络提炼并表示成决策树,将已有问题转化为多
输出回归问题,可以同时获得良好的性能和可解释性。另一方面,知识蒸馏本身也具有一定
的可解释性,从而可以帮助生成可解释的轻量级模型。 Cheng等人
~\citess{Cheng2020ExplainingKD}通过对深度神经网络中间层的量化和分析,提出了对知
识蒸馏所得学生模型性能优越的解释,认为知识蒸馏能够学习到更多的视觉概念。此
外,Phuong等人~\citess{Phuong2019TowardsUK}通过研究线性和深度线性分类器,提出了
蒸馏成功的三个关键原因:一是数据分布的几何特性,二是优化偏差,梯度下降优化找到对
蒸馏目标的非常有利的极小值,三是强单调性,即当训练集的大小增加时,学生分类器的预
期风险总是降低。


{\kaishu 目前知识蒸馏技术已被广泛应用于模型压缩,本项目拟利用部分知识蒸馏技术对
深度学习模型可解释性的提升,融合Kolmogorov-Smirnov检验、D检验等方法构建具有可解
释性的测试覆盖指标,总结归纳模型错误行为的原因,指导训练数据集扩充和模型优化。}








% 因为写 demo,我把参考文献放这里了,真写本子的时候,还是要放在国内外概况那边
\begin{spacing}{1.3} % 行距
	\zihao{5} \songti
	\bibliographystyle{gbt7714-nsfc}
	\bibliography{ref,cai_refs}
	\vspace{11bp}
\end{spacing}
