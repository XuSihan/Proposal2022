\documentclass[a4paper,zihao=-4]{article}
\usepackage[UTF8,punct,linespread=1.56]{ctex}
% \documentclass[a4paper,cs4size,UTF8,punct,linespread=1.56]{ctexart}
\pagestyle{empty} % 第二页以后页码空白
\usepackage[a4paper, left = 3.2cm, right = 3.2cm, top = 2.54cm, bottom = 2.54cm]{geometry}
\usepackage{xcolor}
% \usepackage[citebordercolor = white]{hyperref}
\usepackage[hidelinks]{hyperref}
\usepackage{graphicx}
\usepackage{amsmath}
\usepackage{amssymb}
\usepackage{bm}
\usepackage{times}
\usepackage[subrefformat=parens,labelformat=parens]{subfig} %
\usepackage{booktabs} % for \toprule \midrule \bottomrule \cmidrule
\usepackage{cleveref}
\usepackage{multirow}
\usepackage{url}
\crefformat{table}{表~#2#1#3}
\crefformat{figure}{图~#2#1#3}
\crefformat{equation}{式~(#2#1#3)}
\usepackage{enumitem}
% \setenumerate[1]{itemsep = 0pt, parsep = 0pt, topsep = 2bp}
\setlist[enumerate]{itemsep = 0pt, parsep = 0pt, topsep = 2bp}
% \setitemize[1]{itemsep = 0pt, parsep = 0pt, topsep = 2bp}
\setlist[itemize]{itemsep = 0pt, parsep = 0pt, topsep = 2bp}
\usepackage{fontspec}
\setmainfont{Times New Roman}
% \usepackage{minted}   % For syntax highlighting
% \usemintedstyle{friendly}
\newcommand{\CC}[1]{\cellcolor{gray!#1}}
\usepackage{setspace}
\usepackage{caption}
\DeclareCaptionFont{capfont}{\kaishu\zihao{-4}\selectfont} % Caption font
\DeclareCaptionFont{subfont}{\kaishu\zihao{5}\selectfont} % Sub-caption font
\captionsetup{font = capfont}
\captionsetup[subfigure]{font = subfont}
\captionsetup[figure]{labelsep=space} % 空格 space;点 period;冒号 colon
\captionsetup[table]{labelsep=space}  % 空格 space;点 period;冒号 colon
\usepackage[square,numbers,sort&compress]{natbib}   % For Reference
\newcommand{\citess}[1]{\textsuperscript{\cite{#1}}}
\setlength{\bibsep}{1pt plus 0.3ex}
\usepackage{titlesec}
%\titleformat{\subsubsection}[block]{\hspace{3em}}{\thesubsubsection}{1em}{}
\usepackage{insfc}
\usepackage{enumitem}
\usepackage{verbatim}
\usepackage{colortbl}
\usepackage{makecell}
%\usepackage{titlesec}

%\titleformat{\subsection}[block]{\indent \bfseries}{\arabic{section}.\arabic{subsection}}{1em}{}[]
\titleformat{\subsubsection}[block]{\hspace{2em}\normalsize\bfseries}{(\arabic{subsubsection})}{0.5em}{\vspace{-0.5em}}[]
\titlespacing*{\subsubsection}
{0pt}{1.5ex plus 0.7ex minus .2ex}{1.5ex plus .2ex}

\graphicspath{{images/}}   % 设置图片所存放的目录

\begin{document}

%\kaishu

% Decrease space above and below equations
\setlength{\abovedisplayskip}{0pt}
\setlength{\belowdisplayskip}{0pt}

%%%%%%%%% TITLE %%%%%%%%%
% \title{报告正文 \vspace{-3.4ex}}
% \title{报告正文}
% \maketitle
\begin{center}
	{\kaishu \zihao{3} \textbf{报告正文} \vspace{-3ex}}
\end{center}

\thispagestyle{empty}    % 首页页码空白

%%%%%%%%% Your Content %%%%%%%%%
{\kaishu \zihao{4}参照以下提纲撰写,要求内容翔实、清晰,层次分明,标题突出。}\alert{请勿删除或改动下述提纲标题及括号中的文字。\vspace{9bp}}

\NsfcChapter{(一)立项依据与研究内容}{(建议8000字以内):}

\NsfcSection{1}{项目的立项依据}{(研究意义、国内外研究现状及发展动态分析,需结合科学研究发展趋势来论述科学意义;或结合国民经济和社会发展中迫切需要解决的关键科技问题来论述其应用前景。附主要参考文献目录);}

% !TEX root=./proposal.tex

\subsection{研究意义}


%0.人工智能技术很广泛,智能软件越来越多进入人们的生活1.于此同时,其安全问题也得
%到越来越多的关注2.当前的研究主要针对单个智能组件研究其脆弱性,如对抗攻击,后门
%攻击等。近年来,逐渐有少量研究深度学习基础包和依赖库的漏洞,然而如何利用第三方
%开源组件漏洞去激活智能组件的脆弱性研究较少。
%举例
%挑战:3点
%本项目的研究意义3点

%
% 0.人工智能技术很广泛,智能软件越来越多进入人们的生活
近年来,在数据和算力的驱动下,深度学习技术取得了巨大的成功,逐渐从实验室走向实际
应用,如计算机视觉\citess{dai2021up}、语音识别\citess{baevski2021unsupervised}和
机器翻译\citess{fan2021beyond}等,并开始部署在自动驾驶\citess{feng2021review}、
智慧医疗\citess{liang2021accurate}和航空航天\citess{julian2019deep}等关键任务
上。在获得巨大成功的同时,深度学习模型的错误行为也导致了很多安全事故。2018年3月
19日,在美国亚利桑那州,一辆处在自动驾驶状态的Uber撞击一名女子,致其不幸身亡,同
年7月,Uber宣布停止研发自动驾驶货车\textsuperscript{\cite{Uber}}。这些事故发生的
根本原因是真实世界可能存在各种各样的模型输入,难以遍历所有条件对模型进行测试。深
度学习模型的错误行为不仅为应用本身埋下了隐患,也阻碍了深度学习技术在安全攸关任务
(如\textbf{交通、医疗、军工}等)的应用。因此,{迫切需要在深度学习模型部署前准确
评估其性能,找到其潜在的错误行为来预防未知风险}。\textbf{习总书记指出,要加强人工
智能发展的潜在风险研判和防范,维护人民利益和国家安全,确保人工智能安全、可靠、可
控}。




%深度学习系统错误的巨大危害催生了学术界和工业界众多检测方法的提出与研究。工业界以DeepMind和特斯拉为代表的众多公司针对深度学习模型

深度学习模型错误行为带来的巨大危害催生了众多关于深度学习模型的测试方法的研究。
\textbf{在国外},Goodfellow等人\citess{Odena2019TensorFuzz}首次提出采用模糊测
试的方法变异输入数据,以寻找不符合预期功能的错误行为。加拿大阿尔伯塔大学的Lei Ma
团队深入研究了深度学习模型的测试覆盖指标\citess{ma2018deepgauge,ma2019deepct}和
测试数据生成\citess{xie2019coverage,xie2019deephunter}等问题,将传统软件的边界测
试、组合测试等技术应用到深度学习模型上。\textbf{在国内},南京大学的陈振宇老师团
队首次提出了针对循环神经网络的系统性测试方法\citess{DeepState2022},并在对话系
统、司法文书等应用上提出了新的测试集扩充方法
\citess{liu2021dialtest,guo2020taujud}。西安交通大学的沈超老师团队和天津大学的李
晓红老师团队提出了基于搜索的方法来测试深度学习框架,设计了针对模型结构、参数、权
重和输入的变异算子,能够准确检测出包括逻辑错误、程序崩溃和数值错误在内的三种缺陷
\citess{guo2020audee}。在测试数据选择方面,北京大学的郝丹老师团队提出了一种针对
深度学习模型的测试数据选择方法,通过测试子集的输出概率分布模拟大规模测试集的分
布,估算模型性能\citess{zhou2020cost}。\textbf{面向深度学习模型的测试方法在学术
界取得了显著成效,已成为软件工程领域的新兴研究热点。}

\iffalse
    深度学习的目标可定义为训练一个模型${f}$,使得该模型能够适用于真实数据分布
    $\mathcal D_{gt}$中任意一个从未见过的数据。为了提高真实部署的可靠性,需要系统测
    试深度学习模型$\gamma_{gt}$:$\mathbb{E}_{(x, y) \sim \mathcal{D}_{g t}}
        \mathbb{I}[f(x)=y]$。然而,由于客观世界的真实数据分布是未知的,因此通常在测试集
    $\mathcal D_{\text{test}}$上评估模型性能$\gamma_{\text {test }}:\left(1
        /\left|\mathcal D_{\text {test }}\right|\right) \sum_{(x, y) \in \mathcal
            D_{\text {test }}} \mathbb{I}[f(x)=y]$。因此,\textbf{针对深度学习模型的测试目
        标}为:
    \begin{itemize}
        \item[(1)] 找出使模型做出错误预测的数据$\mathcal D_{\text{failures}}$,即
              $\mathcal D_{\text{failures}}=\{(x, y) | (x, y) \in \mathcal D_{\text{test}}
                  \wedge f(x) \neq y\}$;
        \item[(2)] 生成测试数据集$\mathcal{D}_{\text{test}} \sim \mathcal{D}_{\text{gt}}$
              ,以揭示模型在真实数据分布上所期望的性能$\gamma_{gt}$和实际测试集上所表现的
              性能$\gamma_{\text{test}}$之间的差异;
        \item[(3)] 根据测试反馈信息,找到模型在泛化能力上的不足,进一步提升模型性能。
    \end{itemize}

\fi

\begin{figure}[htp]
    \centering
    \includegraphics[width=0.95\linewidth]{intro.pdf}
    \caption{深度学习模型的测试过程示意}
    \label{fig:ch1:intro}
\end{figure}
%{其可解释性仍旧是一个十分具有挑战性的问题,得到了众多学者的关注和研究}。

\cref{fig:ch1:intro}展示了深度学习模型的一般测试流程。首先,根据测试目标不同,测
试人员会选取不同的测试指标,并基于该测试指标选择或生成对应的测试数据,形成测试
集,输入到深度学习模型,验证模型性能或找到与预期不符的错误行为。\textbf{虽然,现
有测试方法能够在一定程度上反映模型质量或者暴露模型的错误行为,但是,深度学习模型
的测试面临着可解释性不足这一显著问题。}一般来说,可解释性被定义为“向人类解释或呈
现可理解的术语的能力\citess{doshi2017towards}”。在深度学习模型测试的场景中,可解
释性是指向测试人员呈现可以理解的测试反馈的能力。

现有针对深度学习模型的测试方法面临着可解释性不足的问题。具体而言,\textbf{现有方
法无法对测试结果给出细致的、具有明确语义的解释,测试人员无法得知测试成功或失效的
原因,难以根据测试结果来调试和诊断模型},会产生“为什么这个输入数据会导致模型预测
错误?”的疑问。此外,现有测试集生成方法也不具备解释性,\textbf{测试人员无法将测
试集的验证能力与深度学习模型的待测功能联系起来},会产生“为什么这个测试集足够衡量
模型质量?”的疑问,而用户也很难在不理解测试数据选择方法的前提下仅凭借准确率等单
一性能指标而对模型产生足够的信任。因此,\textbf{面向深度学习模型的可解释测试研究
具有十分重要的意义,同时也是人工智能可解释性研究和可信软件研究的重要组成部分}。















%美国白宫颁布了《维护美国在人工智能领域领导地位》、《国家人工智能研发战略》;欧
%盟致力于打造“从实验室进入市场”,发布《2021人工智能协调计划审查》;俄罗斯发布
%《2030年前国家人工智能发展战略》; 2017年7月,我国国务院印发《新一代人工智能发
%展规划》,旨在构筑我国人工智能发展的先发优势,2019年科技部印发《国家新一代人工
%智能创新发展试验区建设工作指引》,全面提升人工智能创新能力和水平。



%2019年国家新一代人工智能治理专业委员会发布《新一代人工智能治理原则——发展负责任
%的人工智能》,该文件中指出“\textbf{人工智能系统应不断提升透明性、可解释性、可靠
%性、可控性},逐步实现可审核、可监督、可追溯、可信赖”。

% 2. 现有工作和方法的不足



\subsection{国内外相关工作}\label{relatedwork}

% 根据与本项目的相关性,本节从覆盖充分性指标、神经网络测试数据生成、深度学习框架测试以及第三方组件漏洞挖掘四个方面介绍和分析国内外研究现状。

根据与本项目的相关性,本节从可解释性研究、测试度量指标、测试集生成和三个方面介绍
和分析国内外相关工作。

\subsubsection{可解释性研究}
{人工智能的可解释性研究是最近几年人工智能领域和软件工程领域(可信人工智能)的研
	究热点}。浙江大学的纪守领老师团队对于目前机器学习和深度学习模型的可解释性研究进
行了详细的调研和总结,调研表明\textbf{目前人工智能的可解释性研究仍旧处于初级阶
	段,并且学术界对深度学习模型的可解释性仍旧缺乏统一的认识}\citess{jishouling}。模
型可解释性总体上可分为\textbf{事前可解释性和事后可解释性}。其中,事前可
解释性指通过训练结构简单、可解释性强的模型或将可解释性结合到模型结构中,使模型本
身具备可解释能力。事后可解释是指通过可解释性方法来对已经训练好的模型进行解释
\citess{jishouling}。

事前可解释性无需额外的信息就可以理解模型的决策过程或决策依据,通常采用结构简单、
易于理解的模型来实现,例如线性回归、决策树、朴素贝叶斯等。事前可解释性要求模型的
决策过程或决策依据可模拟,模型的每一个输入以及每一维特征都有直观的解释
~\citess{lou2012intelligible}。对于朴素贝叶斯模型的预测,可以很容易地转化为单个
特征值的贡献~\citess{strumbelj2010efficient};对于线性模型,其权重则直接反映了特
征重要性~\citess{ribeiro2016should};对于决策树模型,每一条决策路径所对应的条件
规则都为最终的分类结果提供了解释~\citess{huysmans2011empirical}。为了提高简单线
性模型的准确率,同时保留其事前可解释性,广义加性模型通过简单的线性函数组合每个单
特征模型得到最终决策~\citess{wood2006generalized}。另外,事前可解释性也可通过引
入注意力机制来达到,注意力权重矩阵直接体现了模型在决策过程中感兴趣的区域,从而对
模型进行一定程度的解释~\citess{zhang2021context}。

事后可解释性发生在模型训练之后,通过构建解释方法对模型的工作机制、决策行为和决策
依据进行解释。早期针对事后可解释性的研究主要通过规则提取得到对复杂模型整体决策逻
辑的理解~\citess{bastani2017interpreting}。然而,规则提取只能提供近似解释,不一
定能反映待解释模型的真实行为,且受规则复杂度的制约。当模型的结构过于复杂时,要想
从整体上理解模型的决策逻辑通常是很困难的,解决该问题的一个有效途径是降低待解释模
型的复杂度,而\textbf{知识蒸馏(knowledge distillation)则是降低模型复杂度,提高模型可解释
	性的有效方法~\citess{gou2021knowledge}}。{由于知识蒸馏可以完成从教师模型到学生模
型的知识迁移,因而学生模型可以看作是教师模型的全局近似,在一定程度上反映了教师模
型的整体逻辑,因此可以基于学生模型,提供对教师模型的全局解释}。此外,由于整体解
释模型行为较为困难,部分研究以输入样本为导向,通过分析输入样本的每一维特征对模型
最终决策结果的贡献来实现事后可解释性,典型的方法包括敏感性分析解释
~\citess{robnik2008explaining}、局部近似解释~\citess{ribeiro2018anchors}、梯度反
向传播解释~\citess{ding2017visualizing}、特征反演解释
~\citess{dosovitskiy2016inverting}等。然而,现有解释方法很难保证解释结果的准确性
和一致性,为此,Chu等人~\citess{chu2018exact}提出了一种解释方法,可为分段线性神
经网络家族模型提供精确一致的解释,但该方法只能解释线性神经网络模型,无法应用于解
释非线性神经网络模型。

\textbf{尽管已经有很多针对机器学习和深度学习模型本身的可解释性研究,但是对于本项目关注的针对深度学习模型测试的可解释性研
	究工作还十分有限。}Kim等人\citess{Kim2019Guiding}提出一种基于“意外度”的测试覆盖指标,通过测试数据与训练集在隐藏
层表示上的距离,估算测试数据导致模型预测错误的可能性,以此对测试充分性进行可解释分析。Xie等人\citess{Xie2021NPC}
提出了神经元路径覆盖指标,基于神经网络中决策图的控制流和数据流,提出了两种路径覆盖的变体来衡量测试集的充分性。虽
然这些方法在一定程度上提供了针对深度学习模型的测试覆盖指标的解释,但他们只局限于模型推理时神经元的取值,对模型决
策行为和决策依据的解释不足,测试人员难以根据测试结果判断模型是否在以符合人类认知的形式正常工作,也无法确认测试集
对模型决策行为的覆盖性。\textbf{因此,可解释性方法的能力在深度学习模型测试领域并没有得到准确的说明和研究}。

\iffalse
	\textbf{尽管已经有很多针对机器学习和深度学习模型本身的可解释性研究,但是对于本项
		目关注的针对深度学习系统测试方面的可解释性研究工作还十分有限。}Kim等人
	\citess{Kim2019Guiding}提出一种基于“意外度”的测试覆盖指标,通过度量测试数据与训
	练集的不同距离,评估测试集对样本输入空间的覆盖度,从而对测试目标进行一定的可解释
	性分析。Xie等人\citess{Xie2021NPC}提出了神经元路径覆盖指标,类似于传统的程序控制
	流图,首先从深度神经网络中提取决策图用来表示模型的决策逻辑,然后基于决策图的控制
	流和数据流,该方法提出了两种路径覆盖的变体来衡量测试数据在执行决策逻辑时的充分
	性。该测试方法在一定程度上反映出模型的决策逻辑,但由于模型本身缺乏可解释性,难以
	从控制流或数据流路径上辅助开发人员找到模型失效的原因,从而帮助修复模型。但是该方
	法依赖于模型推理过程中的神经元取值,缺乏对模型决策行为的解释,难以检查模型是否在
	以符合人类认知的形式正常工作。
\fi

\subsubsection{测试度量指标}
{测试覆盖指标是衡量测试集充分性的标尺,是软件测试的重要研究问题之一。}对于传统软件,若测试集遍历了待测软件所有的语句、分
支和路径,则在一定程度上表明了测试集对待测软件进行了充分性测试\citess{hilton2018large}。对于深度神经网络而言,其本身的高
维连续特性导致测试集很难遍历所有可能的输入空间,为提高测试集多样性,目前有很多研究提出了关于深度神经网络的结构覆盖指标来
指导测试数据生成,主要分为基于\textbf{单个神经元取值和多个神经元取值组合}两类。

%,从不同角度测试衡量测试集对模型的覆盖充分性。


\iffalse
	\cref{tab:coverage_criteria}总结了现有神经网络测试覆盖指标,其中$m$表示训练
	集规模,$n$表示神经元个数,$l$表示神经网络层数。根据覆盖思想的不同,可分为以下六
	类:

	\begin{table}[htp]
		\renewcommand\arraystretch{1.5}
		\small
		\centering
		\caption{现有神经网络覆盖充分性指标}
		\label{tab:coverage_criteria}
		% \begin{tabular}{p{3cm}p{5cm}p{1cm}p{1cm}p{2cm}}
		\begin{tabular}{ccccc}
			\toprule
			\textbf{序号} & \textbf{主要思想}    & \textbf{覆盖指标}               & \textbf{复杂度} & \textbf{文献号}                                \\
			\midrule
			1             & 基于单个神经元取值   & 神经元覆盖、$k$-多区间覆盖      & $O(n)$          & \cite{ma2018deepgauge}\cite{Pei2019DeepXplore} \\
			2             & 训练集神经元边界     & 神经元边界覆盖、强激活覆盖      & $O(nm)$         & \cite{ma2018deepgauge}                         \\
			3             & 与训练数据分布的距离 & 意外覆盖,平均偏差等            & $O(nm)$         & \cite{Kim2019Guiding}\cite{Tian2019Testing}    \\
			4             & 神经元激活通路覆盖   & 符号-符号覆盖、距离-符号覆盖等  & $O(nl)$         & \cite{Wang2019DeepPath}\cite{Sun2018Testing}   \\
			5             & 神经元的状态转换     & 状态级别覆盖、转换级别覆盖      & $O(n)$          & \cite{Du2018DeepCruiser}                       \\
			6             & 神经元组合测试       & $t$-way组合稀疏覆盖、密集覆盖等 & $O(n^2)$        & \cite{ma2019deepct}                            \\
			\bottomrule
		\end{tabular}
	\end{table}

\fi



%其中涉及到的覆盖指标主要有神经元覆盖率(NC),$k$区间覆盖率(KMNC),重要性驱动覆盖
%(IDC),神经元边界覆盖率(NBC),强神经元覆盖率(SNC),重要神经元覆盖(INC),意外覆
%盖(SC),神经元激活向量距离(NAVD),平均偏差(MD),重要神经元通路覆盖率(INPC),强
%激活通路覆盖率(SAPC),符号-符号覆盖(SSC),距离-符号覆盖(DSC),符号-值覆盖
%(SVC),距离-值覆盖(DVC),状态级别覆盖(BSC),转换级别覆盖(BTC),$t-way$组合稀疏
%%%%覆盖($t-way$ CSC),$t-way$组合密集覆盖($t-way$
%CDC),$(p,t)$完整性覆盖($(p,t)$ C)等指标。

%DeepXplore~\cite{Pei2019DeepXplore}、DeepGauge~\citess{ma2018deepgauge}、IDC~\citess{Gerasimou2020Importance}
%等工作主要围绕神经元覆盖率、K区间覆盖率、重要性驱动覆盖等测试覆盖指标。


{基于单个神经元取值},早期针对深度学习模型的测试覆盖指标以神经元为基本单位来评估
测试充分性。Pei等人\citess{Pei2019DeepXplore}首次提出了针对深度学习模型的白盒测试
指标,即神经元覆盖。该方法将取值高于阈值的神经元视为被激活,通过计算模型中被激活神经元
的比例作为测试覆盖率。 Ma等人\citess{ma2018deepgauge}提出了$k$-多区间覆盖指标,
基于每个神经元在训练集上的取值范围,将其划分为多个取值区间,并计算测试集对这些取值区间
的覆盖率。此外,他们还提出了神经元边界覆盖指标和强激活覆盖指标,统计在测试集中神经元的值超过训练
集上下边界的比例。Gerasimou等人\citess{Gerasimou2020Importance}衡量每个
神经元对模型分类结果的影响,提出了基于神经元重要性的测试覆盖标准。

由于单个神经元对模型内部活动状态的描述存在局限性,目前基于多个神经元取值组合的工
作
\citess{Kim2019Guiding,Tian2019Testing,Wang2019DeepPath,Sun2018Testing,ma2019deepct,Xie2021NPC}
较多。Kim等人\citess{Kim2019Guiding}提出了基于“意外度”的测试覆盖指标,衡量模型的测试数据和训练集在推理时隐藏层
向量表示之间的距离。Tian等人\citess{Tian2019Testing}在图像分类任务上提出一种基于神经元激活频率的白盒测试框架,用来检测分
类器中的混淆和偏差错误。Wang等人\citess{Wang2019DeepPath}提出了一组路径驱动的测试度量指标,以识别对抗性样本。Sun等人
\citess{Sun2018Testing}将传统的MC/DC覆盖标准应用于深度神经网络,采用梯度搜索的方法生成测试集。Ma等人
\citess{ma2019deepct}将组合测试应用于神经网络,提出了$t$-way组合稀疏覆盖和密集覆盖等指标。Xie等人\citess{Xie2021NPC}基于
神经网络决策图的控制流和数据流,提出路径覆盖指标来衡量测试集的充分性。

\textbf{尽管现有测试覆盖指标在一定程度上反映了测试数据的多样性,但面临可解释性不足这
	一问题}。无论是基于单个神经元取值还是多个神经元的取值组合,现有测试度量指标研究集
中于对神经元取值的覆盖程度,测试人员无法将测试度量指标和模型内在的功能联系起来,
无法确认测试覆盖率是否反映了测试集对深度学习模型功能的测试充分性。
\textbf{因为缺少对抽象语义表示覆盖的研究,现有覆盖指标对测试充分性的度量缺乏可解释性。}

%现有指标伸缩性较差,其计算时间和指标有效性难以应对实际大规模模型。本项目拟提出
%基于知识萃取的可解释测试覆盖指标,将知识蒸馏和知识回顾应用于神经网络模型测试,
%从而提升测试指标的伸缩性和可解释性。


%,导致在这些指标引导下的测试数据集生成缺乏符合人类认知的测试目的,难以帮助测试人员诊断和调试模型。

\subsubsection{测试数据集生成}

为了提高深度学习模型的可靠性,使用足够的测试数据对其一般行为和各种边界条件下的极端行为进行充分测试是十分必要的。在深度学
习模型测试研究中,如何生成具有代表性的,容易暴露模型潜在错误行为的测试数据已成为深度学习测试的一个研究重点。关于深度学习
模型测试集生成的研究工作可分为\textbf{测试数据生成和优选}两方面。


\begin{table}[htp]
	\renewcommand\arraystretch{1.5}
	\small
	\centering
	\caption{面向深度学习模型的测试数据生成方法}
	\label{tab:testingDataGen}
	% \begin{tabular}{cp{5cm}p{2cm}cp{2cm}}
	\begin{tabular}{cccccc}
		\toprule
		\textbf{序号} & \textbf{算法思想} & \textbf{评价方法}              & \textbf{测试数据} & \textbf{文献号}                                                             \\
		\midrule
		1             & 模糊测试          & 测试覆盖率、效率               & 图像、文本        & \cite{Odena2019TensorFuzz}\cite{Guo2018DLFuzz}\cite{xie2019coverage}        \\
		2             & 符号执行          & 测试覆盖率、像素重要性等       & 图像、代码        & \cite{Gopinath2018Symbolic}\cite{Sun2018Concolic}                           \\
		3             & 对抗攻击          & 准确率、失真度、人类对比评价等 & 图像、文本等      & \cite{Xiao2018Spatially}\cite{Wicker2018FeatureGuided}\cite{He2018Decision} \\
		\bottomrule
	\end{tabular}
\end{table}


在测试数据生成方面,现有方法可分为基于模糊测试、符号执行和对抗攻击这三类方法(见\cref{tab:testingDataGen})。其中,基于
模糊测试的方法通过随机或者特定规则将输入种子进行变异,观察模型在边界条件下是否会发生错误
\citess{Guo2018DLFuzz,Odena2019TensorFuzz,xie2019coverage,zhang2018deeproad}。Sun等人\citess{Sun2018Concolic}在测试覆盖
指标的基础上,结合具体执行和符号分析生成新的测试数据,以快速提高神经网络的测试覆盖率。Gopinath等人
\citess{Gopinath2018Symbolic}提出了一种轻量级的符号执行技术来测试图像分类模型,解决了重要像素的识别以及创建1像素和2像素
攻击等关键问题。基于对抗样本攻击的方法,很多研究通过向原始样本添加微小扰动的方式生成对抗样本,使深度学习模型做出错误预测
\citess{Xiao2018Spatially,He2018Decision,Wicker2018FeatureGuided}。

\begin{table}[htp]
	\renewcommand\arraystretch{1.5}
	\small
	\centering
	\caption{面向深度学习模型的测试数据优选方法}
	\label{tab:testingDataPri}
	\begin{tabular}{cccc}
		\toprule
		\textbf{序号} & \textbf{算法思想} & \textbf{测试对象} & \textbf{文献号}                                                                                                          \\
		\midrule
		1             & 变异测试方法      & 图像              & \cite{Wang2021Prioritizing}\cite{Ma2018DeepMutation}  \cite{Liu2022DeepState}                                            \\
		2             & 测试数据输出概率  & 图像、自然语言    & \cite{Byun2019Input}\cite{Shen2020MultipleBoundary}\cite{Feng2020DeepGini}\cite{Hu2022AnEmpirical}\cite{Gao2022Adaptive} \\
		\bottomrule
	\end{tabular}
\end{table}

在测试数据优选方面,由于深度学习模型的输入样本空间通常较大,而人工标注测试预言的成本较高,因此很难在系统部署前检测每个输
入样本的正确性。为了解决这个问题,部分研究工作从大规模无标注数据中选择一个测试子集来优先标注和测
试。\cref{tab:testingDataPri}总结了现有的测试数据选择方法,根据算法的思想可分为基于变异测试和模型输出概率的方法。其中,
变异测试通过设计针对深度学习模型的变异算子,从模型结构、参数、权重和输入等方面对模型进行变异,为每个测试数据生成多个模型
变异体,衡量测试数据对模型错误的检测能力,优选能够检测出较多变异体的测试数据来标注和测试
\citess{Ma2018DeepMutation,Wang2021Prioritizing,liu2019exploiting,Liu2022DeepState}。{基于模型的输出概率},部分研究工作
从置信度、不确定性、意外度、敏感度等角度估算测试数据的错误概率,优选可能导致模型预测错误的测试数据进行标注和测试
\citess{Feng2020DeepGini,Hu2022AnEmpirical}。

\textbf{在基于深度学习模型的安全攸关任务上,测试数据集生成方法DeepSuite\citess{xu2021deepsuite}对自动驾驶软件的转向
	(角)预测模型的测试进行了探索性尝试和研究(申请人作为该文章的第一作者,且文章已收录于IEEE T-ITS,中科院SCI一区期刊)}。该方
法主要基于多目标搜索对测试集种子进行多粒度变异和搜索,生成规模可控的具有高代表性的测试集,并在伯克利
DeepDrive(BDD)数据集和Udacity数据集上证明了该方法所生成的测试集具有高代表性和错误检测能力。\textbf{该研究工作说明申请
	人在该领域有坚实的研究基础},本项目将会继续沿着该学术方向在以下方面进行更为深入的研究:
\begin{itemize}
	\item \textbf{现有测试度量指标依赖于对网络结构和神经元值的覆盖,缺乏对模型决策行为和决策依据的理解}。与传统软件
	      测试不同,测试人员无法将测试度量指标和模型的待测功能联系起来,因此,虽然在现有度量指标的引导下能够生成导致模型预
	      测错误的测试数据,但测试人员无法根据测试结果反馈进行模型诊断和调试,导致测试方法与实际应用存在一定的距离。
	\item \textbf{现有测试方法存在面对大规模深度学习模型时伸缩性较差的共性问题,复杂的网络结构导致测试可解释性无法得
		      到更为深入的研究}。天津大学刘爽老师团队的综述论文\citess{survey}对深度神经网络测试研究进行了详细的调研和总
	      结,研究表明目前深度神经网络的测试过程需要较大的计算资源,如何提升对大规模深度神经网络的测试和理解,是一项
	      亟待解决的问题。
	\item \textbf{现有测试场景主要集中于白盒测试场景,而对受限场景下深度学习模型的黑盒测试研究较少,更缺乏针对此种情
		      况的可解释性研究}。现有测试方法假设测试人员能够掌握所有的训练数据和整个深度学习模型,但在许多场景
	      下,例如模型为商业公司私有或第三方机构提供时,测试者无法访问训练数据和模型内部结构,如何在黑盒测试
	      场景下对深度学习模型进行可解释测试,具有重要意义。
	\item \textbf{现有的测试集生成方法缺乏具有解释性的测试逻辑,难以提供能够被测试人员和用户所理解的整体检测报告}。面向深
	      度学习模型的系统性测试方法,不仅要评估模型在给定测试集上的性能表现,还需要建立测试结果与待测模型行为之间的
	      内在联系。现有的测试集生成方法缺乏能够被测试人员和用户所理解的测试逻辑,测试人员无法解释测试集的充分性和代
	      表性,用户很难凭借在给定测试集上的测试结果对模型产生足够的信任,针对深度学习模型的自动化可解释性
	      测试亟待研究。
\end{itemize}




% 因为写 demo,我把参考文献放这里了,真写本子的时候,还是要放在国内外概况那边
\begin{spacing}{1.3} % 行距
	\zihao{5} \songti
	\bibliographystyle{gbt7714-nsfc}
	\bibliography{ref,cai_refs}
	\vspace{11bp}
\end{spacing}


\NsfcSection{2}{项目的研究内容、研究目标,以及拟解决的关键科学问题}{(此部分为重
	点阐述内容);}

\subsection{研究内容}\label{ch2content}

本项目面向医疗卫生行业的数据分析需求,针对电子医疗记录分析存在的研究队列识别困
难、记录时间不规则、模型解释匮乏等问题,以深度学习为基础手段,研究电子医疗记录分
析建模的理论和方法,力争构建端到端的电子医疗记录分析方案,突破队列识别、EMR插补
和可解释性分析模型等关键技术,并在基于电子医疗记录的临床任务上验证本项目的研究成
果。

项目研究工作从队列识别、EMR插补、可解释分析模型和临床任务验证四个层次展开,本项
目的挑战、科学问题和研究内容关系如图~\ref{fig:ch2:rc}所示。各部分研究内容具体介
绍如下:

\begin{figure}
    \begin{small}
        \begin{center}
            \includegraphics[width=0.9\textwidth]{overview.pdf}
        \end{center}
        \caption{挑战、科学问题和研究内容关系图}
        \label{fig:ch2:rc}
    \end{small}
\end{figure}

\subsubsection{基于决策行为模拟的黑盒测试}

借鉴传统软件的路径覆盖测试方法,许多研究提出针对深度学习模型的覆盖性测试方法(参
见\ref{relatedwork} 国内外研究现状及发展动态分析),这些测试覆盖指标和方法主要是
针对白盒的场景,即测试者掌握所有的训练数据和整个深度学习模型,\textbf{但在许多场
景中,测试者无法访问训练数据和模型内部结构,但仍需要对模型的泛化能力进行测试,即
黑盒测试,如深度学习模型是某个公司私有的或者由第三方机构提供,他们只提供了接口或
者打包的可执行程序。}因此,本项目面向黑盒测试场景,研究基于决策行为模拟的黑盒测
试方法。

以分类模型为例,给定一个黑盒深度学习模型$\mathcal M$和测试数据集$\mathcal
D=\{x^{(i)},y^{(i)}\}$,测试者可以得到模型针对每个输入$x^{(i)}$(input)的输出
$\hat{y}^{(i)}$,通常是该输入属于各个类别的概率分布,除此之外,测试者无法知道该
模型的训练数据和内部结构。可见,深度学习模型黑盒测试面临两个主要问题:
\begin{itemize}
    \item \textbf{如何建立模型有关的测试方法?}仅有的几个黑盒模型测试方法仅针对
    数据的多样性进行评估,无法准确反映模型的泛化能力,也不能给模型改进提供有效建
    议。
    \item \textbf{如何掌握模型的决策机制?}黑盒测试无法了解模型的结构,难以掌握模型的决策机制。
\end{itemize}

\begin{figure}
    \begin{small}
        \begin{center}
            \includegraphics[width=0.8\textwidth]{ch2_2Btest.pdf}
        \end{center}
        \caption{黑盒测试研究内容}
        \label{fig:ch2:2Btest}
    \end{small}
\end{figure}
本项目深度学习模型黑盒测试的主要研究内容如\cref{fig:ch2:2Btest}所示,\textbf{本项目首先
研究针对黑盒模型的预测行为模拟方法,拟利用知识萃取的方法,通过建立副本模型
$\mathcal M^\prime$(如决策树)萃取黑盒模型$\mathcal M$的知识,模拟$\mathcal M$
的预测行为;然后,在副本模型的基础上,本项目拟建立基于决策路径覆盖度的测试方法,
通过覆盖度指标反映模型的决策机制和泛化能力。}

\subsubsection{融合医学偏差的EMR自动插补}

患者就医时间不规律,且不同医疗特征的记录频率也差别很大,导致电子医疗记录具有时间
不规则性和稀疏性的特点,为电子医疗记录分析带来了新的挑战。本项目拟研究通用的电子
医疗记录插补方法,对电子医疗记录进行预处理,使其在时间维度上更为规则,降低后续分
析的模型复杂度。

\begin{figure}
    \begin{small}
        \begin{center}
            \includegraphics[width=0.85\textwidth]{ch2imputation.pdf}
        \end{center}
        \caption{EMR插补研究内容}
        \label{fig:ch2:imputation}
    \end{small}
\end{figure}

图~\ref{fig:ch2:imputation}举例说明了本项目电子医疗记录插补的研究内容。给定电子
医疗记录$\bm X$,若电子医疗记录一共有$d$个特征,$t$个时间点,则$\bm X \in
\mathbb{R}^{d\times t}$。本项目拟构建机器学习模型推断$\bm X$中的部分缺失值,即图
中\textit{NA}所在的位置,最终得到数值完整的电子医疗记录,如图中$\bm X_{ipt}$所
示。

现有工作利用RNN、Transformer~\citess{vaswani2017attention}和它们的变种针对时间维
度进行建模,但现有工作没有考虑电子医疗记录产生过程引入的医学偏差,导致部分插补不
够准确。电子医疗记录不能直接反映患者的健康状态,因为这些记录的产生除了源自患者的
健康状态,还蕴含着患者与电子病历系统的“交互”过程,通常会引入系统偏差。医学研究表
明~\citess{agniel2018biases},\textbf{医学偏差不应被看作数据质量问题或者噪声,而
是一种数据细分的信号},如:肾衰竭的患者更可能在晚上10点到早上6点之间化验肌酐;重
病患者做检查更加频繁等。记录产生的时间可以反映特征记录的过程,是模型理解医学偏差
的重要途经。如图~\ref{fig:ch2:imputation}所示,本项目拟结合患者的缺失标记矩阵和
就医时间学习特征缺失规律,在EMR插补模型中引入医学偏差。若时间$(k)$时患者第$i$个
特征被记录下来,则图中缺失标记矩阵$\bm M$的对应取值$m_i^{(k)}=1$,若没观察到,则
$m_i^{(k)}=0$,缺失矩阵和记录时间完整地反映了医疗特征的缺失规律,可用于理解医学
偏差。\textbf{本项目拟研究如何利用缺失矩阵和记录时间为EMR插补模型引入医学偏差,
以实现隐式的数据细分,提升EMR插补的准确性和合理性}。

\subsubsection{结合特征重要性和时间关联性的可解释预测模型}

如图~\ref{fig:ch2:interpretability}所示,电子医疗记录除了包含随时间变化的医疗特
征,还会包含患者的人口统计学特征,如性别、种族等,这些特征虽然是静态不变的,但对
预测患者病情发展也非常重要,所以\textbf{在构建预测模型时,首先应该研究如何将患者
的静态特征和动态特征分别建模,以及如何融合两组特征进行分析}。

\begin{figure}
    \begin{small}
        \begin{center}
            \includegraphics[width=0.75\textwidth]{ch2Interpretability.pdf}
        \end{center}
        \caption{可解释预测模型研究内容}
        \label{fig:ch2:interpretability}
    \end{small}
\end{figure}

模型的临床应用需要提供对模型和预测结果的解释,因此,预测模型的可解释性是本项目另
一个重要的研究内容。目前用于电子医疗记录分析的可解释方法可分为两类:第一类研究在
预测模型建模时没有考虑其解释性,但利用后解释(post-interpretable)模型来解释预测
结果,这类方法具有两个缺点,一是后解释模型把任何预测模型当作黑盒,依然无法让医生
了解模型真实的决策过程,只能提供预测结果的可靠程度,二是随着就诊数据的变化,需要
重新训练预测模型和后解释模型,带来了额外开销;第二类研究在预测模型中实现可解释
性,在完成预测的同时,也能为预测结果提供可解释的依据,这类方法尝试在预测准确性和
可解释性两方面找到一种平衡,通常在模型实现可解释性时,其准确率会有所下降。

本项目拟在预测模型中考虑可解释性,现有工作缺少对静态特征和动态特征的统一比较,无
法得到模型的全局可解释性。静态特征通常用浅层全连接层建模,本项目拟采用逐层相关性
传播~\citess{bach2015pixel}分析静态特征与预测结果的关系。而对动态特征,本项目拟
采用RNN模型进行建模,但由于RNN复用隐含层,其针对多维特征的可解释特别差,一方面,
隐含层是由所有特征经过多步计算得到的,无法从隐含层解耦出每个特征对预测结果的贡
献;另一方面,RNN中最后的隐含层是由多步计算得到,其时间信息也无法逆向得到,阻碍
了特征出现时间的分析。所以动态特征可解释性研究的重点在于分析高维的动态特征与预测
目标的关系。\textbf{本项目通过对RNN隐含层解耦,挖掘动态特征重要性和时间关联性,
并统一比较动态特征和静态特征的贡献,实现模型的全局解释,且可回溯样本预测结果,为
医务人员提供全面的模型解释}。

\subsubsection{临床预测性任务}

\textbf{本项目拟将研究成果应用于具体的临床预测性任务,以验证其有效性,同时,通过
和临床医生的合作,提高研究质量,加速研究成果应用}。具体而言,本项目拟将研究成果
应用于慢性肾脏病进展预测和急诊科感染性休克预警,并通过和医生的交流,保证项目研究
内容具有实际应用价值。

\begin{itemize}
    \item  \textbf{与北京大学医疗健康大数据国家研究院合作,预测DKD慢性肾病患者的
    病情发展情况},提醒医生及时进行临床干预。
    \item  利用机器学习手段预测病人状态对降低ICU感染性休克的病发率和死亡率具有重
    要意义。\textbf{本项目拟基于MIMIC-III~\citess{johnson2016mimic},利用项目研
    究成果实现ICU患者感染性休克预警系统},验证项目研究成果的有效性。
\end{itemize}


\subsection{研究目标}\label{ch2target}

本项目从医疗卫生行业和人工智能结合的实际需求出发,针对队列识别泛化能力差、EMR插
补准确率低、以及预测模型可解释性匮乏等问题,研究队列识别、EMR插补和可解释预测模
型等问题,并将相关研究成果应用实际临床数据,检验本项目的研究成果在电子医疗记录预
测性分析中的应用效果,推动电子医疗记录分析的研究进展。

具体研究目标包括:

\textbf{在技术方面},本项目拟在以下四方面实现技术突破: (a) 提出基于表现型的队列
识别方法,充分利用已标注和未标注数据,提高队列识别模型的准确性和表示学习的泛化能
力;(b) 提出融合医学偏差的EMR自动插补方法,实现数据预处理,通过合理引入医学偏
差,提高EMR插补的准确性;(c) 提出可解释预测模型,在进行预测的同时,可以从模型中
得到与预测目标相关的特征和时间点;(d) 将本项目的研究成果应用于DKD患者病情进展预
测和重症监护室患者感染性休克预测。

\textbf{在成果形式方面},本项目力争在国内外高水平期刊、会议上发表论文8篇以上,全
部被SCI/EI检索,其中有重要影响的论文4篇以上;申请专利2项。

\textbf{在人才培养方面},通过本项目研究,培养深度学习、电子医疗记录分析、可解释人工智能等交叉领域的青年人才,拟培养研究生3-4人。

\subsection{拟解决的关键科学问题}

基于上述研究目标和研究内容,本项目拟在以下几个关键理论和技术问题上有所突破:

\subsubsection{在弱监督条件下构建泛化性强的患者层次表示,实现自动队列识别}

从海量、高维的电子医疗记录中识别研究队列是电子医疗记录分析的基础,由于电子医疗记
录的高维性和表现型的组合性,依靠医务人员人力筛选和标注所有符合预定表现型的队列数
据不具有可操作性。患者的表现型表示可以更准确地对应预定义表现型,提高队列识别准确
度,但仅基于少量队列标签的EMR数据集不足以让模型学习到好的数据分布,同时,传统半
监督方法在标注数据和未标注数据上分别训练,模型的训练更容易被未标注数据影响。因
此,如何合理利用大量未标注数据辅助建模,构建泛化性强的患者层次表示,实现电子医疗
记录的队列识别是本项目拟解决一个关键科学问题。

\subsubsection{建立医学偏差和缺失矩阵的对应关系,并将其融入电子医疗记录插补中}

电子医疗记录具有时间不规则性,直接在电子医疗记录上建模和训练难以得到较好的结果,
训练得到的模型通常容易过拟合,泛化性较差。在构建预测模型之前对电子医疗记录进行插
补具有重要意义,可以降低预测模型复杂度和训练难度。现有方法忽略了电子医疗记录中固
有的医学偏差,将其中的缺失值当作完全随机缺失对待,其插补结果不够准确。医学偏差并
不是数据质量问题或者噪声,反而对充分理解电子医疗记录非常重要。如何建立医学偏差和
记录缺失的对应关系,并将其融合到电子医疗记录插补中是本项目构建通用电子医疗记录处
理框架要解决的一个关键科学问题。

\subsubsection{融合特征重要性和时间关联性,建立电子医疗记录可解释分析模型}

模型的可解释性对医学领域的预测性分析至关重要,决定了模型是否能应用于临床实践。电
子医疗记录(动态特征)是一种特殊的多元时序数据,通常采用RNN模型建模,但在RNN学习
过程中,多个特征在不同时间点的取值都被糅合在RNN的隐含层中,无法总结得到特征对于
模型的意义,也无法计算出某条记录和预测结果的相关性。因此,如何融合特征重要性和时
间关联性,构建可解释预测模型,支持模型的全局解释和预测结果的回溯,也是本项目拟解
决的一个关键科学问题。


\NsfcSection{3}{拟采取的研究方案及可行性分析}{(包括研究方法、技术路线、实验手段、关键技术等说明);}

\subsection{研究方案和技术路线}

\begin{figure}[h]
    \begin{small}
        \begin{center}
            \includegraphics[width=0.95\textwidth]{ch3solution.pdf}
        \end{center}
        \caption{总体技术路线}
        \label{fig:ch3:solution}
    \end{small}
\end{figure}

围绕\ref{ch2content}规划的研究内容和\ref{ch2target}制定的研究目标,本项目拟定的
总体技术路线如\cref{fig:ch3:solution}所示。本项目针对电子医疗记录分析构建基于深
度学习的端到端的解决方案,通过对电子医疗记录数据集依次进行队列识别、EMR插补和可
解释预测,实现对特定患者队列的分析挖掘。本项目的主要研究内容分别对应大数据分析通
用流程中的数据获取、数据预处理和数据分析。此外,鉴于医疗领域的特殊性,本项目的研
究工作需要与医生进行充分的沟通,在模型设计和结果解释等方面融入医生的反馈。下面针
对各部分研究内容,详细介绍其具体研究方案和技术路线。



\subsubsection{基于决策行为模拟的黑盒测试}\label{ch3_1}

\cref{fig:ch3:2Btest}展示了本项目基于决策行为模拟的黑盒测试研究方法,如图所示,
\textbf{本项目拟利用知识蒸馏技术萃取黑盒模型中的知识}。就申请人所知,目前尚未有
基于知识蒸馏的深度学习覆盖度测试研究。首先,本项目利用知识蒸馏技术从黑盒模型(``
教师''模型)中萃取知识,训练一个小模型(``学生''模型)模拟其预测行为,在知识蒸馏
中,``教师''模型被视为黑盒模型,正好契合本项目黑盒测试的场景;其次,本项目拟设计
针对小模型的测试方法,间接评估黑盒模型的泛化能力,一方面,蒸馏得到的``学生''模型
复杂度低,可有效解决深度学习模型测试计算开销大的问题,另一方面,``学生''模型的内
部结构可以自定义,测试者也可访问,本项目拟利用基于树的模型(如:决策树、随机森林
等),以提高测试结果的可解释性。

具体地,给定一个黑盒模型$\mathcal M$,本项目将$\mathcal M$视为``教师''模型,假定
对于任意的输入$x^{(i)}$,测试者仅能得到$\mathcal M$对于该输入预测的概率分布,即
$x^{(i)}$属于各类的概率,记为$p^{(i)}$。$(x^{(i)}, p^{(i)})$的对应关系即为模型
$\mathcal M$中蕴含的知识,将作为训练``学生''模型时的软目标。\textbf{本项目拟构建
基于树的模型作为``学生''模型,记为$\mathcal M_t$,以支持可解释模型测试}。根据知
识蒸馏的思想,在训练``学生''模型时,本项目融合软目标$(x^{(i)}, p^{(i)})$和硬目标
$(x^{(i)}, y^{(i)})$构建``学生''模型的损失函数,使树型模型的输出同时接近黑盒模型
的概率分布$p^{(i)}$和真实标签$y^{(i)}$。值得注意的是,在训练``学生''模型时,直接
使用``教师''模型SoftMax层的输出结果$p^{(i)}$不合适,因为小模型无法直接学习得到大
模型的效果,我们通过在``学生''模型的损失函数中引入知识蒸馏中的T(Temperature)参
数,放大分类错误的误差,缩小正确分类的误差,可有效提高``学生''模型训练的效果。

\begin{figure}[htp]
    \begin{small}
        \begin{center}
            \includegraphics[width=0.9\textwidth]{ch3_2Btest.pdf}
        \end{center}
        \caption{基于决策行为模拟的黑盒测试研究方案}
        \label{fig:ch3:2Btest}
    \end{small}
\end{figure}

\textbf{本项目拟针对``学生''模型建立基于决策路径的覆盖性测试方法},知识蒸馏得到
的``学生''模型$\mathcal M_t$的预测行为非常接近原黑盒模型$\mathcal M$,针对
$\mathcal M_t$的覆盖度测试能比较准确地反映$\mathcal M$的泛化能力。首先,从基于树
的``学生''模型$\mathcal M_t$中抽取该模型对每个测试样本的决策路径,得到路径集合
$\mathcal P=\{t_1, t_2,\dots, t_l\}$,$l$表示路径数。然后,\textbf{本项目拟利用
统计方法建立基于决策路径的覆盖度指标},以评估测试用例集的充分性,其基本思想为对
树模型的决策路径覆盖度高的测试用例集具有良好的充分性,在此基础上,本项目拟计算决
策路径覆盖频率和正态分布的拟合程度来评估测试用例集的分布,可采用
Kolmogorov-Smirnov检验、D检验等方法,并结合正太性检验方法的拟合优度和决策路径的
覆盖度作为测试用例集充分性评价指标。此外,\textbf{本项目拟利用树模型的可解释性,
总结归纳模型错误行为的原因,指导训练数据集扩充和模型优化}。

\subsubsection{基于层次语义理解的白盒测试}\label{ch3_2}

本项目针对白盒测试的研究方案首先利用知识蒸馏技术,训练可解释的``学生''模型。以
\cref{fig:ch3:WBtestKD}为例,\textbf{本项目拟首先从给定的白盒模型中抽取其预测行
为},具体而言,本项目拟采线性判别分析(Linear Discriminative Analysis, LDA)抽取
各层样本表示的关键特征,同时可实现降维的效果,有效提高后续分析计算的效率,然后,
利用自底向上的层次聚类方法,对LDA得到的关键样本特征进行聚类,理清每一组神经网络
层的判别决策能力。本项目认为深度学习模型对输入的表示学习是一个从粗粒度到细粒度的
过程,因此,浅层表示(如第1组的输出)没有能力将样本分为最终指定的类别,仅能做到
粗粒度分类,但层次聚类算法会以打到最大类别数为目标,所以\textbf{本项目拟分析中间
层表示的整体分布,修改层次聚类的优化目标,或者在层次聚类之后,再合并相近的簇,以
准确反映中间层的决策语义}。

另一方面,\textbf{得到各组(层)的决策语义后,本项目拟改进知识蒸馏的方式,将各层
的决策语义融入``学生''模型的优化目标中},其主要思想如公式\eqref{eq:kd}所示:
\begin{equation}
    \mathcal{L} = \mathcal{L}_o + \beta\sum_{k=1}^L KL(p(f_s^k(\bm x_i)), p(g^k(y_i))) \,,
    \label{eq:kd}
\end{equation}
其中$\mathcal L_o$是``学生''模型原有的损失函数,$f_s^k(\bm x_i)$表示``学生''模型
第$k$层的表示向量,$g^k(y_i)$将数据标签$y_i$转换成第$k$层的语义标签,本项目拟用
KL散度来度量``学生''模型表示向量的分布是否和抽取的决策语义接近,作为损失函数的正
则项,引导``学生''模型训练。

\begin{figure}[htp]
    \begin{small}
        \begin{center}
            \includegraphics[width=0.95\textwidth]{ch3_WBtestKD.pdf}
        \end{center}
        \caption{基于层次语义理解的知识蒸馏研究方案}
        \label{fig:ch3:WBtestKD}
    \end{small}
\end{figure}

得到可解释的``学生''模型后,本项目拟基于该模型提出决策路径覆盖度测试指标。根据公
式\eqref{eq:kd},在测试阶段,容易得到各层输出所属的粗粒度类别,\textbf{如
\cref{fig:ch3:WBtest}所示,本项目拟分析测试用例集对该决策路径图的覆盖程度,用以
评估测试用例集的充分性,同时,可检测``学生''模型针对某样本的分类过程是否有异常决
策路径(图中虚线所示),以分析模型在不同测试用例上的泛化能力},具有异常决策路径
的测试用例可能是边界样本,很可能引发模型错误,是模型进一步优化提供重要参考。
\begin{figure}[htp]
    \begin{small}
        \begin{center}
            \includegraphics[width=0.65\textwidth]{ch3_WBtest.pdf}
        \end{center}
        \caption{基于可解释``学生''模型的覆盖度测试研究方案}
        \label{fig:ch3:WBtest}
    \end{small}
\end{figure}

\subsubsection{基于反馈偏置的自适应测试集生成}\label{ch3_3}

为了平衡测试集的规模和质量,针对给定大规模无标注数据,本项目拟在有限标注成本空间
内生成高质量测试集,尽可能模拟真实的测试数据分布,在\textbf{可控制规模内}生成具
有高代表性和检测能力的测试集,针对深度学习模型进行可解释测试。由于深度学习模型缺
乏可解释性且现有覆盖指标伸缩性差,本项目基于知识抽取方法模拟决策路径,将具有相同
决策路径的测试数据作为同类,采用\textbf{聚类分析和最大平均差异法}( Maximum Mean
Discrepancy,MMD)选取与大规模无标注测试数据具有近似决策路径分布的代表性数据作为
测试集,并反馈已选测试数据及其测试结果反馈,自适应选择测试数据。


\begin{figure}[htp]
    \begin{small}
        \begin{center}
            \includegraphics[width=0.85\textwidth]{ch3interpretability.pdf}
        \end{center}
        \caption{可解释预测模型(动态特征)研究方案}
        \label{fig:ch3:interpretability}
    \end{small}
\end{figure}

\cref{fig:ch3:interpretability}是本项目针对自适应测试集生成研究方案的示意图。首
先,为了测试的伸缩性和可解释性,采用知识蒸馏/知识回顾的方法抽取复杂神经网络的决
策路径,针对每个测试数据都获得其可解释性决策路径。以图中描述的模拟决策路径$t_i$
为例,每条决策路径对应$n_i$个无标注测试数据,视为一组。我们对每组测试数据进行聚
类分析,根据每组数据的规模将其聚为$k_i$类,然后利用最大平均差异的方法从每类测试
数据中选取$m$个测试数据,若该类数少于$m$个则全部选取,构成初始测试子集。最大差异
法可表示为以下公式:
\begin{equation}
    \begin{aligned}
        \operatorname{MMD}(\mathcal{F}, P, G)=\sup _{f \in \mathcal{F}}\left(\mathbb{E}_{X \sim P}[f(X)]-\mathbb{E}_{Y \sim G}[f(Y)]\right)
    \end{aligned}
\end{equation}
其中$\bm X= (\bm x_1, \bm x_2, \dots, \bm x_t)$为模型输入,$z$是隐变量,表示数据
特征的编号,$z \in {1,2,\dots, v}$,用于解释不同特征的重要性和时间关联性。
$I_{T_{1i}} \cdot \bm h_{1i}, I_{T_{2i}} \cdot \bm h_{2i}, \dots, I_{T_{ti}}
\cdot \bm h_{ti}$即为特征$i$时间维度的注意力,而$P(z=i|I_{F_1} \cdot \bm h_{t1},
I_{F_2} \cdot \bm h_{t2}, \dots, I_{F_v} \cdot \bm h_{tv})$则为特征$i$的全局重要
性。

反馈偏置为

{本项目拟从充分性测试和准确
评估模型两个测试目标来制定自适应反馈机制,将复杂神经网络抽象成可解释蒸馏模型,选
取与大规模无标注测试数据具有近似决策路径分布的代表性数据进行标注,为测试人员提供
全阶段的可解释测试方法}。

子集可经过可解释蒸馏模型得到所有测试数据的决策路径分布,具有相同决策路径的测试数
据为一组。根据每组测试数据规模大小,采用聚类分析将具有相同决策路径的测试数据分为
多个类簇,利用基于最大平均差异法为从每簇测试数据中选取固定总数的测试数据作为代表
性数据,得到与无标注数据集有近似决策路径分布的初始子集

\subsection{可行性分析}

\subsubsection{理论可行性}

本项目研究目标明确,研究内容清晰,研究方案和技术路线中所应用的方法和技术手段在业
界都有着成熟清晰的理论基础。申请人和项目组对这些关键技术和理论有着深入的了解和掌
握,近年来在人工智能、医疗数据分析等领域的高水平会议上发表了多篇论文。项目组前期
已经对本项目中提到的研究内容分别进行了详细的调研和分析,并在医疗特征表示学习、时
间序列插补等领域取得了初步成果。因此,从理论上说,本项目是可行的。

\subsubsection{技术可行性}

申请人前期调研了大量基于深度学习的电子医疗记录分析的研究工作(如
\ref{relatedwork}节所述),深度学习模型不仅能挖掘高维特征之间复杂的关系,同时能
有效处理数据中长时间依赖关系,非常适合电子医疗记录分析。利用深度学习技术解决电子
医疗记录预测性分析需要解决三个核心问题,即队列识别、EMR不规则性处理和模型的可解
释性,申请人之前的研究工作一直聚焦于电子医疗记录的分析,在医疗特征表示学习、多元
时序数据插补等方面已取得一定的研究成果,并在攻读博士期间与医生合作,参与过医疗分
析系统的开发,相关研发经验可作为本项目的研究基础。申请人所在项目组多年来一直活跃
在在大数据处理和分析、机器学习等领域,在相关领域有着丰富的研究经验和技术积累。同
时,在研究过程中,申请人与医疗机构建立了合作关系,积累了大量可用于实验的真实数据
集。因此,从技术上说,本项目是可行的。

\subsubsection{团队合理性}

项目组在大数据分析和人工智能领域具有一定的基础,积累了丰富的研究经验,在重要国际
会议上发表了多篇高水平论文,项目在信息系统和数据分析系统开发方面也有丰富的积累,
可为本项目可视化分析系统研发提供保障。项目组梯队完善,队伍具有凝聚力和创造力,项
目组成员每周定期讨论,有着良好的科研氛围,同时对本项目的研究内容具有浓厚的研究兴
趣。申请人与联合培养时的导师新加坡国立大学教授Beng Chin Ooi(黄铭钧)一直保持密
切联系,Ooi教授长期研究大数据管理与分析,是数据库和数据挖掘方面非常活跃的科学
家,可为本项目提供技术指导。

申请人与医疗机构和专家一直保持良好的合作关系,与北京大学医疗健康大数据国家研究院
张路霞教授合作研究慢性肾病患者病情进展,和天津市肿瘤医院合作研究肺癌患者治疗效果
评估,同时,申请人参与建设“南开大学-基准联合医学研究中心”,和广州基准医疗有限公
司合作研究基于多模态医疗数据的癌症早诊技术。这些合作单位和专家可为本项目提供专业
意见的反馈,保证研究内容符合医学常识和临床需求。综上所述,项目团队组成合理,能保
障本项目的顺利完成。

\NsfcSection{4}{本项目的特色与创新之处;}{}

与现有研究工作相比,本项目的特色与创新之处体现在以下几方面:

\begin{itemize}
    \item[(1)] \textbf{本项目提出的针对深度学习模型的可解释测试研究本身具有创新
          性}。赋予人工智能算法可解释研究以具体的应用场景,在多种测试应用场景
          下,对可解释性进行深入研究。{深度学习模型的可解释测试研究,对设计更具
          有测试逻辑的测试目标,解释测试集的充分性和代表性,理解测试执行失效的原
          因,以及辅助诊断和调试模型都具有重要意义。推动用户更好地信任深度学习模
          型的测试结果。}该项研究将会成为人工智能可解释性和可信软件工程的重要组
          成部分,对可解释性研究产生非常积极的作用。
    \item[(2)] 针对现有测试覆盖指标缺乏可解释性、伸缩性差的共性问题,\textbf{提
              出基于层次语义理解的测试覆盖目标抽取方法,构建基于层次语义逻辑的覆
              盖性测试方法,可以填补可解释测试领域国内外研究空白}。现有白盒测试
              方法主要依赖于对神经元的值和取值组合的覆盖,由于深度学习模型包含较
              多神经元,\textbf{现有测试覆盖指标计算复杂度高,不具备可解释性,缺
              乏实际应用价值}。因此,本项目利用知识回顾的方法,分层抽取深度学习
              模型中的决策语义,并将抽取得到的多层语义信息融合到知识蒸馏中,进一
              步提高白盒测试的伸缩性,以适用于较大的深度学习模型。该研究可显著提
              高深度学习白盒测试的可解释性和伸缩性,是本项目相比于现有白盒测试方
              法的重要创新之处。
    \item[(2)] \textbf{针对受限场景下测试者无法访问模型内部结构的问题,本项目研
              究基于知识蒸馏的决策行为模拟方法,提高黑盒测试的有效性和可解释
              性}。目前针对深度学习模型黑盒测试的研究比较缺乏,仅有的研究是对与
              模型无关的测试集多样性分析,缺乏对模型测试行为的建模和理解,一方面
              难以提供对模型决策逻辑和决策依据的测试覆盖信息,测试人员无法准确判
              断测试充分性和代表性;另一方面,现有方法无法为模型开发提供有效的测
              试反馈信息。因此,在黑盒测试场景中,本项目利用知识蒸馏,从模型预测
              的概率分布出发,学习得到一个可解释的学生模型,通过全局近似模拟黑盒
              模型的决策行为,构建伸缩性强、可解释的黑盒覆盖测试指标,是本项目一
              大特色和创新之处。

    \item[(4)] 针对测试数据选取和测试结果反馈缺乏可解释性的问题,\textbf{本项目
              在可解释测试覆盖指标的基础上,研究融合基于实例的解释方法和反馈偏
              置,自适应地持续选取测试数据}。现有测试集生成方法的研究缺乏可解释
              性,导致测试人员针对深度学习模型的测试执行和测试数据选取缺乏理解,
              难以有效辅助诊断和调试模型。本项目在可解释黑盒测试和白盒测试的基础
              上,得到测试数据的决策路径分布,融合基于实例的解释方法和反馈偏置选
              取代表数据和边界数据,生成具有可解释性的测试集。\textbf{项目研究成
              果可为测试人员提供单个测试数据执行失效的原因,同时也提供整个测试集
              选取的解释方法,相比较于现有测试方法更具有可解释性和实用性,可提高
              深度学习模型测试的可信度}。
\end{itemize}

\NsfcSection{5}{年度研究计划及预期研究结果}{(包括拟组织的重要学术交流活动、国际合作与交流计划等)。}

\subsection{年度研究计划}
本项目实施3年,具体年限为2023年1月1日至2025年12月31日。具体计划如下:

\textbf{第一阶段(2023年1月--2023年12月)},重点研究基于决策行为模拟的深度神经网
络黑盒测试方法,实现利用副本模型进行知识萃取的技术,并提出适用于复杂神经网络的测
试覆盖指标,具体包括:
\begin{itemize}[itemindent=2em]
    \item[(1)] 研究基于知识萃取的决策行为模拟方法;
    \item[(2)] 提出适用于复杂神经网络的具有可解释性的测试覆盖指标; 
    \item[(3)] 申请发明专利1项;
    \item[(4)] 在国际期刊或会议上发表研究论文1-2篇,参加国际会议1-2次,并在会议上
    做论文成果报告。
\end{itemize}

\textbf{第二阶段(2024年1月--2024年12月)},重点研究基于层次语义理解的深度神经网
络白盒测试技术,实现对原复杂模型的决策路径抽取和副本模型训练技术,并提出新的结构
覆盖指标,具体包括:
\begin{itemize}[itemindent=2em]
    \item[(1)] 研究对复杂模型的决策路径抽取和副本模型构建方法;
    \item[(2)] 提出适用于复杂神经网络的白盒测试覆盖指标; 
    \item[(3)] 申请发明专利1项;
    \item[(4)] 在国际期刊或会议上发表研究论文1-2篇,参加国际会议1-2次,并在会议
    上做论文成果报告。
\end{itemize}

\textbf{第三阶段(2025年1月--2025年12月)},重点研究基于反馈偏置的自适应测试集生
成方法,构建测试反馈机制,实现自适应的测试集生成方法,具体包括:
\begin{itemize}[itemindent=2em]
    \item[(1)] 构建神经网络测试反馈机制,支持测试结果和行为特征的关联挖
    掘;
    \item[(2)] 基于白盒/黑盒测试反馈偏置,实现自适应的测试集生成方法;
    \item[(3)] 在国际期刊或会议上发表研究论文1-2篇,参加国际会议1-2次,并在会议上
    做论文成果报告; 
    \item[(4)] 项目总结,完成结项报告,准备验收。
\end{itemize}

\subsection{预期研究成果}

本项目的预期研究成果包括以下几个方面:
\begin{itemize}[itemindent=2em]
    \item[(1)] 在CCF-A类推荐期刊/会议或其他SCI上发表高水平论文3-6篇;;
    \item[(2)] 申请专利2项;
    \item[(3)] 培养研究生2-3人; 
    \item[(4)] 完成一个面向复杂神经网络的可解释测试框架;
    \item[(5)] 开源相关研究工作,供用户下载,并提供说明和使用文档。
\end{itemize}


%%%%%%%%%%%%%%%%%%%%%%%%%%%%%%%%%%%%%%%%%%%%%%%%%
\NsfcChapter{(二)研究基础与工作条件}{}


\NsfcSection{1}{研究基础}{(与本项目相关的研究工作积累和已取得的研究工作成绩);}

本项目申请人自攻读博士学位以来一直从事智能软件的质量评估和测试分析相关研究,主要
研究方向是深度学习测试、智能软件工程、开源软件分析等,具有较强的理论基础,且掌握
先进的实用技术。{申请人在国内外重要会议期刊上发表学术论文十余篇,其中发表
TOSEM、IEEE T-ITS以及计算机学报等CCF-A类或SCI一区期刊论文5篇,CCF-B类会议或期刊2
篇}。在新加坡国立大学联合培养期间,曾参与智能软件安全质量项目GEMS,\textbf{以
{\underline{第一作者}在软件可靠性领域旗舰会议ISSRE 2017上发表论文并入选{Best
Paper Session}}},相关工作被开发成Eclipse插件,已有1500余次下载。在模型可解释性
应用领域,提出一种可解释的开源软件许可证兼容性检测模型,研究成果\textbf{以
\underline{第一作者}发表于软件工程领域顶刊TOSEM 2022},并开源相关数据集和代码。
此外,{为提高深度学习技术部署在自动驾驶领域的可靠性},提出\textbf{一种基于基于多
目标搜索的自动驾驶软件白盒测试方法},研究成果\textbf{以\underline{第一作者}发表
于智能交通系统领域顶刊IEEE T-ITS 2021},与国汽智控(北京)科技有限公司、中汽研
(天津)汽车工程研究院有限公司紧密合作,相关研究成果运用于自动驾驶计算基础平台并
取得良好反馈。

\subsection{部分相关研究工作}


申请人前期与本项目相关的部分研究工作:

\begin{itemize}
	\item \textbf{在深度学习测试方向,从泛化能力的角度评估了现有测试度量指标在衡量样本多样性、检测模型错误和异常样本等方面
		      的能力,并在此基础上提出了一种面向基于深度学习的自动驾驶系统白盒测试方法}。研究发现大部分基于神经元结构覆盖的
	      测试指标与分布内(In-distribution)和分布外(Out-of-distribution)的错误样本相关性较小,错误检测能力较低,且
	      \textbf{随着模型复杂度的增加,测试度量指标的错误检测能力明显降低,而测试消耗的时间和空间资源显著增加}。在此基
	      础上提出了基于数据分布差异的深度学习模型白盒测试方法,生成与训练数据分布差异大、错误检测能力强、规模相对较小的
	      测试数据集。该基础工作推动了本项目在深度学习测试方向的开展与实施,部分成果已发表于国际顶级期刊\underline{T-ITS
		      2021(SCI一区},参见已发表论文[1])上,并与中汽研(天津)汽车工程研究院有限公司、国汽智控(北京)科技有限公司
	      紧密合作,相关研究成果运用于自动驾驶计算基础平台并取得良好反馈。

	\item \textbf{在深度学习应用可解释性方向,深入研究了深度学习在源代码表示、许可证分析、自然语言接口等类型的输入数据上的
		      应用,构建了具有可解释性的深度学习模型。}在基于神经网络的代码表示模型上,通过注意力机制和代码语义特征抽取,构
	      建可解释的智能软件重构模型,推荐函数抽取、重命名等重构方法并解释重构原因;在许可证兼容性检测模型上,结合语法分
	      析和命名实体识别,自动抽取许可证条款并进行极性分析,预测许可证兼容性并给出不兼容原因。这些基础工作使得申请人掌
	      握了深度学习模型可解释性相关的理论和研究方法,为本项目提出深度学习可解释性测试方法提供了坚实的基础。相关研究成
	      果已发表于软件工程领域顶级期刊\underline{TOSEM 2022(CCF-A,SCI一区},参见已发表论文[2])、计算机研究与发展
	      (\underline{中文CCF-A},参见已发表论文[4])以及软件可靠性领域顶级会议\underline{ISSRE 2017(CCF-B},参见已发
	      表论文[3])上。

	\item \textbf{在软件安全与质量方面,研究了基于决策制导的测试数据生成方法和基于相关统计量的缺陷定位方法}。提出了针对访
	      问控制漏洞的测试数据生成方法,该方法在保证覆盖率的同时,精简测试用例,减少了测试集冗余度,设计并实现原型系统
	      ACTCGenerator并运行在公开的Web应用上。针对软件中同时存在多个缺陷导致难以定位的问题,提出了基于缺陷相关统计量的缺
	      陷定位方法,利用统计模型区分不同缺陷导致的软件失效,提高定位效率。这些基础工作使得申请人掌握了关于测试数据生成和
	      缺陷定位与修复的相关理论和技术,其思想可辅助本项目推动测试覆盖指标和测试数据生成方法的构建,为测试基于深度学习的
	      智能软件提供了坚实的基础。相关研究工作于发表于计算机学报(\underline{中文CCF-A},参见已发表论文[6])。
\end{itemize}




\subsection{已发表的相关论文}

\begin{enumerate}[label={[\arabic*]}]
	\item \underline{Sihan Xu}, Zhiyu Wang, Lingling Fan, Xiangrui Cai, Hua Ji,
	      Siau-Cheng Khoo, and Brij Bhooshan Gupta. DeepSuite: A test suite optimizer
	      for autonomous vehicles. In IEEE Transactions on Intelligent Transportation
	      Systems, 2021. doi: 10.1109/TITS.2021.3131808. (\textbf{SCI一区})
	
	\item \underline{Sihan Xu}, Ya Gao, Lingling Fan, Zheli Liu, Yang Liu, and
	      Hua Ji. LiDetector: License incompatibility detection for open source
	      software. In ACM Transactions on Software Engineering and Methodology
	      (TOSEM). 2022. doi:10.1145/3518994. (\textbf{SCI一区, CCF-A})
	\item \underline{Sihan Xu}, Aishwarya Sivaraman, Siau-Cheng Khoo, and Jing
	      Xu. Gems: An extract method refactoring recommender. In Proceedings of the
	      28th International Symposium on Software Reliability Engineering (ISSRE).
	      IEEE, 2017: 24-34. (\textbf{CCF-B})

	      \iffalse
	\item \underline{Sihan Xu}, Ya Gao, Xiangrui Cai, Zhiyu Wang, and Hua Ji.
	      Effective Multi-Fault Localization Based on Fault-Relevant Statistics. In
	      Proceedings of the 45th Annual Computers, Software, and Applications Conference
	      (COMPSAC). IEEE, 2021: 998-1003. (\textbf{CCF-C})

	\item \underline{Sihan Xu}, Sen Zhang, Weijing Wang, Xinya Cao, Chenkai Guo,
	      and Jing Xu. Method name suggestion with hierarchical attention
	      networks[C]//Proceedings of the 2019 ACM SIGPLAN Workshop on Partial
	      Evaluation and Program Manipulation (PEPM). 2019: 10-21. (\textbf{CCF-C})
	      \fi

	\item 潘璇,\underline{徐思涵*},蔡祥睿,温延龙,袁晓洁. 基于深度学习的数据库自然语言接口综述[J]. 计
	算机研究与发展, 2021, 58(9): 1925. (\textbf{中文CCF-A})

	\item Xuesong Zhao, Yanbo Shao, Juanyun Mai, Airu Yin, and \underline{Sihan
		      Xu}. Respiratory sound classification based on BiGRU-attention network with
	      XGBoost. In Proceedings of the 13th International Conference on
	      Bioinformatics and Biomedicine (BIBM). IEEE, 2020: 915-920. (\textbf{CCF-B})

	\item 文硕,许静,苑立英,李晓虹,\underline{徐思涵},司冠南. 基于策略推导的访问控制漏洞测试用例生成方法[J].
	计算机学报, 2017, 40(12): 2658-2670. (\textbf{中文CCF-A})

	\item 过辰楷,许静,司冠南,李恩鹏,\underline{徐思涵}. 面向移动应用软件信息泄露的模型检测研究[J].
	计算机学报, 2016, 39(11): 2324-2343. (\textbf{中文CCF-A})


\end{enumerate}


基于上述工作,申请人有信心达到本项目的研究目标,并取得高水平研究成果。

\NsfcSection{2}{工作条件}{(包括已具备的实验条件,尚缺少的实验条件和拟解决的途径,包括利用国家实验室、国家重点实验室和部门重点实验室等研究基地的计划与落实情况);}

\textbf{经费和硬件条件方面:}本项目依托于天津市网络与数据安全技术重点实验以及天
津市媒体计算技术工程研究中心。项目组已完成多项国家自然科学基金项目和重要省部级项
目,为本项目的研究奠定了坚实的理论基础和丰富的科研经验。通过前期研究课题的积累,
实验室已经建有大数据计算集群,拥有服务器20余台和微机100余台,其中包含IBM、DELL等
品牌高性能服务器,并安装了较齐全的软件系统以及开发工具,其强大的计算性能能够满足
实验所需要的计算条件。本项目尚缺乏1台塔式图形工作站,以便更好地进行深度学习模型
的训练和调试工作,拟将依靠项目经费解决。

\textbf{人员方面:}申请人所在项目组梯队完善,科研氛围浓厚,每周开展讨论班活动,
每两周开展大组会活动,科研创新性与科研工作量均有保证。实验室已培养本领域的博士和
硕士五十余名,在学博士和硕士研究生十余名,在合理的分配下将会有较为充足的硕博研究
生一同参与到本项目的研究中。

\textbf{国内外合作方面:},申请人与英国爱丁堡大学的Ji Hua研究员,新加坡国立大学
的Khoo Siau-Cheng教授,新加坡南洋理工大学Liu Yang教授,天津大学的陈森副教授等国内外
知名专家保持长期合作关系,与本项目组的刘哲理教授、范玲玲副教授等均有密切合作,他
们一直活跃在软件测试、软件安全和模型安全等领域的学术前沿,可为本项目提供技术指
导。

\NsfcSection{3}{正在承担的与本项目相关的科研项目情况}{(申请人正在承担的与本项目相关的科研项目情况,包括国家自然科学基金的项目和国家其他科技计划项目,要注明项目的资助机构、项目类别、批准号、项目名称、获资助金额、起止年月、与本项目的关系及负责的内容等);}

无

\NsfcSection{4}{完成国家自然科学基金项目情况}{(对申请人负责的前一个已资助期满的科学基金项目(项目名称及批准号)完成情况、后续研究进展及与本申请项目的关系加以详细说明。另附该项目的研究工作总结摘要(限500字)和相关成果详细目录)。}

%%%%%%%%%%%%%%%%%%%%%%%%%%%%%%%%%%%%%%%%%%%%%%%%%

无

\NsfcChapter{(三)其他需要说明的问题}{}

\NsfcSection{1}{}{申请人同年申请不同类型的国家自然科学基金项目情况(列明同年申请的其他项目的项目类型、项目名称信息,并说明与本项目之间的区别与联系)。}

无

\NsfcSection{2}{}{具有高级专业技术职务(职称)的申请人是否存在同年申请或者参与申请国家自然科学基金项目的单位不一致的情况;如存在上述情况,列明所涉及人员的姓名,申请或参与申请的其他项目的项目类型、项目名称、单位名称、上述人员在该项目中是申请人还是参与者,并说明单位不一致原因。}

无

\NsfcSection{3}{}{具有高级专业技术职务(职称)的申请人是否存在与正在承担的国家自然科学基金项目的单位不一致的情况;如存在上述情况,列明所涉及人员的姓名,正在承担项目的批准号、项目类型、项目名称、单位名称、起止年月,并说明单位不一致原因。}

无

\NsfcSection{4}{}{其他。}

无


\end{document}
