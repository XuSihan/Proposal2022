% !TEX root=./proposal.tex

\subsection{研究意义}


%0.人工智能技术很广泛,智能软件越来越多进入人们的生活1.于此同时,其安全问题也得
%到越来越多的关注2.当前的研究主要针对单个智能组件研究其脆弱性,如对抗攻击,后门
%攻击等。近年来,逐渐有少量研究深度学习基础包和依赖库的漏洞,然而如何利用第三方
%开源组件漏洞去激活智能组件的脆弱性研究较少。
%举例
%挑战:3点
%本项目的研究意义3点

%
% 0.人工智能技术很广泛,智能软件越来越多进入人们的生活
近年来,在数据和算力的驱动下,深度学习取得了巨大的成功。如\cref{fig:ch1:intro}所
示,深度学习通过特征变换和逐层处理,学习输入数据的特征表示,深度学习模型对复杂问
题的解决能力使其研究成果逐渐从实验室走向实际应用,如计算机视觉
\citess{dai2021up}、语音识别\citess{baevski2021unsupervised}和机器翻译
\citess{fan2021beyond}等,并开始部署在自动驾驶\citess{feng2021review}、智慧医疗
\citess{liang2021accurate}和航空航天\citess{julian2019deep}等关键任务上。在获得巨
大成功的同时,深度学习模型的错误行为也导致了很多安全事故。2018年3月19日,在美国
亚利桑那州,一辆处在自动驾驶状态的Uber撞击一名女子,致其不幸身亡,同年7月,Uber
宣布停止研发自动驾驶货车。这些事故发生的根本原因是真实世界可能存在各种各样的模型
输入,难以在所有条件下测试所有可能的测试输入。深度学习系统的错误行为不仅为应用本
身埋下了隐患,也阻碍了深度学习技术在安全攸关任务(如\textbf{交通、医疗、军工}
等)的应用。因此,\textbf{迫切需要在深度学习模型部署前准确评估其性能,找到潜在的
错误行为来预防未知风险}。%习总书记说



%深度学习系统错误的巨大危害催生了学术界和工业界众多检测方法的提出与研究。工业界以DeepMind和特斯拉为代表的众多公司针对深度学习模型

深度学习模型的错误行为带来的巨大危害催生了众多关于深度学习系统检测方法的提出与研
究。 Goodfellow等人\citess{Odena2019TensorFuzz}首次提出了采用模糊测试的方法变异
输入数据,以寻找不符合预期功能的错误行为。加拿大阿尔伯塔大学的Lei Ma团队深入研究
了神经网络测试覆盖指标\citess{ma2018deepgauge,ma2019deepct}和测试数据生成
\citess{xie2019coverage,xie2019deephunter}等问题,将传统软件的边界测试、组合测
试等技术应用到神经网络上。国内南京大学的陈振宇老师团队首次提出了针对循环神经网络
的系统性测试方法\citess{DeepState2022},并提出了循环神经网络在对话系统、司法文
书等应用上的测试集扩充方法\citess{liu2021dialtest,guo2020taujud}。西安交通大学
的沈超老师团队提出了一种基于搜索的方法测试深度学习框架,设计了针对模型结构、参
数、权重和输入的变异算子,测试出逻辑错误、程序崩溃和数值错误三类缺陷
\citess{guo2020audee}。在测试数据选择方面,北京大学的郝丹老师团队提出了一种针对
深度学习模型的测试数据选择方法,通过测试子集模拟大规模测试集的分布,估算模型性能
\citess{zhou2020cost}。\textbf{面向深度学习系统的测试方法在学术界取得了显著成
效,已成为软件工程领域的新兴研究热点。}

\iffalse

深度学习的目标可定义为训练一个模型${f}$,使得该模型能够适用于真实数据分布
$\mathcal D_{gt}$中任意一个从未见过的数据。为了提高真实部署的可靠性,需要系统测
试深度学习模型$\gamma_{gt}$:$\mathbb{E}_{(x, y) \sim \mathcal{D}_{g t}}
\mathbb{I}[f(x)=y]$。然而,由于客观世界的真实数据分布是未知的,因此通常在测试集
$\mathcal D_{\text{test}}$上评估模型性能$\gamma_{\text {test }}:\left(1
/\left|\mathcal D_{\text {test }}\right|\right) \sum_{(x, y) \in \mathcal
D_{\text {test }}} \mathbb{I}[f(x)=y]$。因此,\textbf{针对深度学习模型的测试目
标}为:
\begin{itemize}
    \item[(1)] 找出使模型做出错误预测的数据$\mathcal D_{\text{failures}}$,即
          $\mathcal D_{\text{failures}}=\{(x, y) | (x, y) \in \mathcal D_{\text{test}}
              \wedge f(x) \neq y\}$;
    \item[(2)] 生成测试数据集$\mathcal{D}_{\text{test}} \sim \mathcal{D}_{\text{gt}}$
          ,以揭示模型在真实数据分布上所期望的性能$\gamma_{gt}$和实际测试集上所表现的
          性能$\gamma_{\text{test}}$之间的差异;
    \item[(3)] 根据测试反馈信息,找到模型在泛化能力上的不足,进一步提升模型性能。
\end{itemize}

\fi

\begin{figure}[htp]
    \centering
    \includegraphics[width=0.9\linewidth]{intro.pdf}
    \caption{面向深度学习系统的测试数据生成与检测}
    \label{fig:ch1:intro}
\end{figure}

如\cref{fig:ch1:intro}所示,面向深度学习系统的测试方法步骤较为明确。在测试覆盖度
量的指引下,生成测试集输入到深度学习模型,比较模型输出与预期是否一致,评估模型性
能,找到与预期不符的模型错误行为。\textbf{虽然,深度学习系统测试的方法能够生成测
试数据并暴露模型错误行为,但是,深度学习系统的测试面临着可解释性不足这一显著问
题。}深度学习技术已经在不同的领域,例如图形图像、自然语言处理、语音识别、自动驾
驶、声纹识别等重要的前言领域取得了显著性效果。但是,{人工智能的可解释性仍旧是一
个十分具有挑战性的问题,得到了众多学者的关注和研究}。在人工智能场景中,可解释性
被定义为“向人类解释或呈现可理解的术语的能力\citess{doshi2017towards}”。深度学习
系统的测试方法同样面临着可解释性不足的问题,\textbf{因为现有测试方法无法对测试结
果给出细致的、具有明确语义的解释,测试人员无法得知测试成功或失效的原因,也不能根
据测试结果定位模型缺陷},测试人员会产生“为什么这个输入数据会导致模型预测错误?”
的疑问。另一方面,现有测试方法所生成的测试集也缺乏可解释性,\textbf{测试人员无法
将测试集与深度学习系统的待测功能联系起来},无法确认测试集对深度学习系统的检测能
力,会产生“为什么这个测试集能够衡量模型质量?”的疑问,而用户也很难凭借在任意测试
集上的准确率等单一性能指标而对模型产生足够的信任。因此,\textbf{进行面向深度学习
系统测试的可解释性研究具有十分重要的意义,同时也是人工智能可解释性研究和可信软件
研究的重要组成部分}。









    
    
    



%美国白宫颁布了《维护美国在人工智能领域领导地位》、《国家人工智能研发战略》;欧
%盟致力于打造“从实验室进入市场”,发布《2021人工智能协调计划审查》;俄罗斯发布
%《2030年前国家人工智能发展战略》; 2017年7月,我国国务院印发《新一代人工智能发
%展规划》,旨在构筑我国人工智能发展的先发优势,2019年科技部印发《国家新一代人工
%智能创新发展试验区建设工作指引》,全面提升人工智能创新能力和水平。



%2019年国家新一代人工智能治理专业委员会发布《新一代人工智能治理原则——发展负责任
%的人工智能》,该文件中指出“\textbf{人工智能系统应不断提升透明性、可解释性、可靠
%性、可控性},逐步实现可审核、可监督、可追溯、可信赖”。

% 2. 现有工作和方法的不足
虽然已有研究在测试度量指标构建、测试输入生成和测试预言生成等方面已取得一定研究成
果,但受传统结构覆盖测试的影响,\textbf{现有研究主要集中于对以神经元为基本单位的
神经网络架构进行结构覆盖测试},从神经网络和传统软件的主要区别以及深度学习模型测
试的实际需求两方面考虑,现有研究主要存在以下问题:





\textbf{(1)覆盖目标缺乏明确语义,导致测试结果难以辅助开发人员}

随着数据和算力的发展,深度学习模型的规模近年来呈越来越大的趋势,很多研究证明了大
规模模型对复杂问题的解决能力~\cite{kenton2019bert,he2021masked}。然而,现有的深
度学习结构覆盖测试指标大多难以适用于大规模深度学习模型,主要体现在以下两个方面:
第一,随着模型复杂度的提高,神经元及其取值组合的数量也随之提高,以神经元为基础结
构的结构覆盖率和测试多样性之间的相关性越来越小,导致结构覆盖率对测试集质量的评估
能力骤减;第二,现有结构覆盖指标的计算复杂度与神经元的个数或训练集的规模呈线性增
长关系,伸缩性较差,难以适用于实际使用的大规模深度学习模型。


\textbf{(2)在受限场景下,无法拿到神经网络结构,如何}

面向深度学习模型的系统性测试,不仅要评估深度学习模型在特定测试集上的性能表现,还
需要建立测试结果与模型行为之间的联系,从而辅助开发人员修复提升模型。由于深度学习
模型的黑盒性质,现有的基于神经元的结构覆盖指标不能体现深度学习系统的决策逻辑,缺
乏有效语义。虽然在测试覆盖指标的引导下可以生成测试数据,甚至暴露模型的错误行为,
但与传统软件不同的是,开发人员无法知道这些测试数据的成功或失效执行的原因,导致测
试反馈结果很难辅助开发人员定位模型缺陷,提升模型性能。

\textbf{(3)测试集的检测能力缺乏可解释性,难以提供具有可解释性的整体检测报告}

测试用例由输入数据和预期输出两部分组成。在测试一个深度学习模型时,其输入数据的样
本空间经常较大,而人工标注测试预言的成本较高,因此很难在系统部署前检测每个潜在数
据样本的预测正确性。比如图片分类任务中,输入数据是高维连续的图像数据,而目前对图
像的自动标注研究尚在起步阶段,通常需要人工标注图片的类别作为测试用例的预期输出,
在执行测试用例后,根据模型的输出是否符合预期来判断测试用例的成功与否。与此同时,
深度学习模型的测试对测试集的质量有较高要求,即一方面期望测试集能够尽可能模拟真实
数据的分布,准确评估模型性能,另一方面尽早发现容易导致模型预测错误的测试数据,从
而避免在模型部署后仍然存在大量意外输入,导致模型决策错误。为了在有限标注成本内提
高测试的有效性,需要提高测试集的效率,生成规模可控、高质量的测试集。


% 4. 研究方法
深度学习技术正在潜移默化地改变人们的工作和生活方式,其在自动驾驶、恶意软件检测以
及飞机碰撞避免系统等安全攸关领域部署的核心在于“质量”,而深度学习模型的测试技术是
保障深度学习模型质量的必要手段,在这一背景下,\textbf{本项目针对深度学习模型测试
存在的伸缩性差、缺乏可解释性以及标注成本过高的问题,研究适用于大规模神经网络、兼顾测
试成本和测试质量的可解释测试技术,解决了深度神经网络的黑盒复杂特性给模型测试带来的
挑战,为深度学习模型部署在安全攸关领域提供核心技术支撑。}


\subsection{国内外相关工作}\label{relatedwork}

% 根据与本项目的相关性,本节从覆盖充分性指标、神经网络测试数据生成、深度学习框架测试以及第三方组件漏洞挖掘四个方面介绍和分析国内外研究现状。

根据与本项目的相关性,本节从可解释性研究、测试度量指标、测试集生成和三个方面介绍
和分析国内外相关工作。

\subsubsection{可解释性研究}
{人工智能的可解释性研究是最近几年人工智能领域和软件工程领域(可信人工智能)的研
	究热点}。浙江大学的纪守领老师团队对于目前机器学习和深度学习模型的可解释性研究进
行了详细的调研和总结,调研表明\textbf{目前人工智能的可解释性研究仍旧处于初级阶
	段,并且学术界对深度学习模型的可解释性仍旧缺乏统一的认识}\citess{jishouling}。模
型可解释性总体上可分为\textbf{事前可解释性和事后可解释性}。其中,事前可
解释性指通过训练结构简单、可解释性强的模型或将可解释性结合到模型结构中,使模型本
身具备可解释能力。事后可解释是指通过可解释性方法来对已经训练好的模型进行解释
\citess{jishouling}。

事前可解释性无需额外的信息就可以理解模型的决策过程或决策依据,通常采用结构简单、
易于理解的模型来实现,例如线性回归、决策树、朴素贝叶斯等。事前可解释性要求模型的
决策过程或决策依据可模拟,模型的每一个输入以及每一维特征都有直观的解释
~\citess{lou2012intelligible}。对于朴素贝叶斯模型的预测,可以很容易地转化为单个
特征值的贡献~\citess{strumbelj2010efficient};对于线性模型,其权重则直接反映了特
征重要性~\citess{ribeiro2016should};对于决策树模型,每一条决策路径所对应的条件
规则都为最终的分类结果提供了解释~\citess{huysmans2011empirical}。为了提高简单线
性模型的准确率,同时保留其事前可解释性,广义加性模型通过简单的线性函数组合每个单
特征模型得到最终决策~\citess{wood2006generalized}。另外,事前可解释性也可通过引
入注意力机制来达到,注意力权重矩阵直接体现了模型在决策过程中感兴趣的区域,从而对
模型进行一定程度的解释~\citess{zhang2021context}。

事后可解释性发生在模型训练之后,通过构建解释方法对模型的工作机制、决策行为和决策
依据进行解释。早期针对事后可解释性的研究主要通过规则提取得到对复杂模型整体决策逻
辑的理解~\citess{bastani2017interpreting}。然而,规则提取只能提供近似解释,不一
定能反映待解释模型的真实行为,且受规则复杂度的制约。当模型的结构过于复杂时,要想
从整体上理解模型的决策逻辑通常是很困难的,解决该问题的一个有效途径是降低待解释模
型的复杂度,而\textbf{知识蒸馏(knowledge distillation)则是降低模型复杂度,提高模型可解释
	性的有效方法~\citess{gou2021knowledge}}。{由于知识蒸馏可以完成从教师模型到学生模
型的知识迁移,因而学生模型可以看作是教师模型的全局近似,在一定程度上反映了教师模
型的整体逻辑,因此可以基于学生模型,提供对教师模型的全局解释}。此外,由于整体解
释模型行为较为困难,部分研究以输入样本为导向,通过分析输入样本的每一维特征对模型
最终决策结果的贡献来实现事后可解释性,典型的方法包括敏感性分析解释
~\citess{robnik2008explaining}、局部近似解释~\citess{ribeiro2018anchors}、梯度反
向传播解释~\citess{ding2017visualizing}、特征反演解释
~\citess{dosovitskiy2016inverting}等。然而,现有解释方法很难保证解释结果的准确性
和一致性,为此,Chu等人~\citess{chu2018exact}提出了一种解释方法,可为分段线性神
经网络家族模型提供精确一致的解释,但该方法只能解释线性神经网络模型,无法应用于解
释非线性神经网络模型。

\textbf{尽管已经有很多针对机器学习和深度学习模型本身的可解释性研究,但是对于本项目关注的针对深度学习模型测试的可解释性研
	究工作还十分有限。}Kim等人\citess{Kim2019Guiding}提出一种基于“意外度”的测试覆盖指标,通过测试数据与训练集在隐藏
层表示上的距离,估算测试数据导致模型预测错误的可能性,以此对测试充分性进行可解释分析。Xie等人\citess{Xie2021NPC}
提出了神经元路径覆盖指标,基于神经网络中决策图的控制流和数据流,提出了两种路径覆盖的变体来衡量测试集的充分性。虽
然这些方法在一定程度上提供了针对深度学习模型的测试覆盖指标的解释,但他们只局限于模型推理时神经元的取值,对模型决
策行为和决策依据的解释不足,测试人员难以根据测试结果判断模型是否在以符合人类认知的形式正常工作,也无法确认测试集
对模型决策行为的覆盖性。\textbf{因此,可解释性方法的能力在深度学习模型测试领域并没有得到准确的说明和研究}。

\iffalse
	\textbf{尽管已经有很多针对机器学习和深度学习模型本身的可解释性研究,但是对于本项
		目关注的针对深度学习系统测试方面的可解释性研究工作还十分有限。}Kim等人
	\citess{Kim2019Guiding}提出一种基于“意外度”的测试覆盖指标,通过度量测试数据与训
	练集的不同距离,评估测试集对样本输入空间的覆盖度,从而对测试目标进行一定的可解释
	性分析。Xie等人\citess{Xie2021NPC}提出了神经元路径覆盖指标,类似于传统的程序控制
	流图,首先从深度神经网络中提取决策图用来表示模型的决策逻辑,然后基于决策图的控制
	流和数据流,该方法提出了两种路径覆盖的变体来衡量测试数据在执行决策逻辑时的充分
	性。该测试方法在一定程度上反映出模型的决策逻辑,但由于模型本身缺乏可解释性,难以
	从控制流或数据流路径上辅助开发人员找到模型失效的原因,从而帮助修复模型。但是该方
	法依赖于模型推理过程中的神经元取值,缺乏对模型决策行为的解释,难以检查模型是否在
	以符合人类认知的形式正常工作。
\fi

\subsubsection{测试度量指标}
{测试覆盖指标是衡量测试集充分性的标尺,是软件测试的重要研究问题之一。}对于传统软件,若测试集遍历了待测软件所有的语句、分
支和路径,则在一定程度上表明了测试集对待测软件进行了充分性测试\citess{hilton2018large}。对于深度神经网络而言,其本身的高
维连续特性导致测试集很难遍历所有可能的输入空间,为提高测试集多样性,目前有很多研究提出了关于深度神经网络的结构覆盖指标来
指导测试数据生成,主要分为基于\textbf{单个神经元取值和多个神经元取值组合}两类。

%,从不同角度测试衡量测试集对模型的覆盖充分性。


\iffalse
	\cref{tab:coverage_criteria}总结了现有神经网络测试覆盖指标,其中$m$表示训练
	集规模,$n$表示神经元个数,$l$表示神经网络层数。根据覆盖思想的不同,可分为以下六
	类:

	\begin{table}[htp]
		\renewcommand\arraystretch{1.5}
		\small
		\centering
		\caption{现有神经网络覆盖充分性指标}
		\label{tab:coverage_criteria}
		% \begin{tabular}{p{3cm}p{5cm}p{1cm}p{1cm}p{2cm}}
		\begin{tabular}{ccccc}
			\toprule
			\textbf{序号} & \textbf{主要思想}    & \textbf{覆盖指标}               & \textbf{复杂度} & \textbf{文献号}                                \\
			\midrule
			1             & 基于单个神经元取值   & 神经元覆盖、$k$-多区间覆盖      & $O(n)$          & \cite{ma2018deepgauge}\cite{Pei2019DeepXplore} \\
			2             & 训练集神经元边界     & 神经元边界覆盖、强激活覆盖      & $O(nm)$         & \cite{ma2018deepgauge}                         \\
			3             & 与训练数据分布的距离 & 意外覆盖,平均偏差等            & $O(nm)$         & \cite{Kim2019Guiding}\cite{Tian2019Testing}    \\
			4             & 神经元激活通路覆盖   & 符号-符号覆盖、距离-符号覆盖等  & $O(nl)$         & \cite{Wang2019DeepPath}\cite{Sun2018Testing}   \\
			5             & 神经元的状态转换     & 状态级别覆盖、转换级别覆盖      & $O(n)$          & \cite{Du2018DeepCruiser}                       \\
			6             & 神经元组合测试       & $t$-way组合稀疏覆盖、密集覆盖等 & $O(n^2)$        & \cite{ma2019deepct}                            \\
			\bottomrule
		\end{tabular}
	\end{table}

\fi



%其中涉及到的覆盖指标主要有神经元覆盖率(NC),$k$区间覆盖率(KMNC),重要性驱动覆盖
%(IDC),神经元边界覆盖率(NBC),强神经元覆盖率(SNC),重要神经元覆盖(INC),意外覆
%盖(SC),神经元激活向量距离(NAVD),平均偏差(MD),重要神经元通路覆盖率(INPC),强
%激活通路覆盖率(SAPC),符号-符号覆盖(SSC),距离-符号覆盖(DSC),符号-值覆盖
%(SVC),距离-值覆盖(DVC),状态级别覆盖(BSC),转换级别覆盖(BTC),$t-way$组合稀疏
%%%%覆盖($t-way$ CSC),$t-way$组合密集覆盖($t-way$
%CDC),$(p,t)$完整性覆盖($(p,t)$ C)等指标。

%DeepXplore~\cite{Pei2019DeepXplore}、DeepGauge~\citess{ma2018deepgauge}、IDC~\citess{Gerasimou2020Importance}
%等工作主要围绕神经元覆盖率、K区间覆盖率、重要性驱动覆盖等测试覆盖指标。


{基于单个神经元取值},早期针对深度学习模型的测试覆盖指标以神经元为基本单位来评估
测试充分性。Pei等人\citess{Pei2019DeepXplore}首次提出了针对深度学习模型的白盒测试
指标,即神经元覆盖。该方法将取值高于阈值的神经元视为被激活,通过计算模型中被激活神经元
的比例作为测试覆盖率。 Ma等人\citess{ma2018deepgauge}提出了$k$-多区间覆盖指标,
基于每个神经元在训练集上的取值范围,将其划分为多个取值区间,并计算测试集对这些取值区间
的覆盖率。此外,他们还提出了神经元边界覆盖指标和强激活覆盖指标,统计在测试集中神经元的值超过训练
集上下边界的比例。Gerasimou等人\citess{Gerasimou2020Importance}衡量每个
神经元对模型分类结果的影响,提出了基于神经元重要性的测试覆盖标准。

由于单个神经元对模型内部活动状态的描述存在局限性,目前基于多个神经元取值组合的工
作
\citess{Kim2019Guiding,Tian2019Testing,Wang2019DeepPath,Sun2018Testing,ma2019deepct,Xie2021NPC}
较多。Kim等人\citess{Kim2019Guiding}提出了基于“意外度”的测试覆盖指标,衡量模型的测试数据和训练集在推理时隐藏层
向量表示之间的距离。Tian等人\citess{Tian2019Testing}在图像分类任务上提出一种基于神经元激活频率的白盒测试框架,用来检测分
类器中的混淆和偏差错误。Wang等人\citess{Wang2019DeepPath}提出了一组路径驱动的测试度量指标,以识别对抗性样本。Sun等人
\citess{Sun2018Testing}将传统的MC/DC覆盖标准应用于深度神经网络,采用梯度搜索的方法生成测试集。Ma等人
\citess{ma2019deepct}将组合测试应用于神经网络,提出了$t$-way组合稀疏覆盖和密集覆盖等指标。Xie等人\citess{Xie2021NPC}基于
神经网络决策图的控制流和数据流,提出路径覆盖指标来衡量测试集的充分性。

\textbf{尽管现有测试覆盖指标在一定程度上反映了测试数据的多样性,但面临可解释性不足这
	一问题}。无论是基于单个神经元取值还是多个神经元的取值组合,现有测试度量指标研究集
中于对神经元取值的覆盖程度,测试人员无法将测试度量指标和模型内在的功能联系起来,
无法确认测试覆盖率是否反映了测试集对深度学习模型功能的测试充分性。
\textbf{因为缺少对抽象语义表示覆盖的研究,现有覆盖指标对测试充分性的度量缺乏可解释性。}

%现有指标伸缩性较差,其计算时间和指标有效性难以应对实际大规模模型。本项目拟提出
%基于知识萃取的可解释测试覆盖指标,将知识蒸馏和知识回顾应用于神经网络模型测试,
%从而提升测试指标的伸缩性和可解释性。


%,导致在这些指标引导下的测试数据集生成缺乏符合人类认知的测试目的,难以帮助测试人员诊断和调试模型。

\subsubsection{测试数据集生成}

为了提高深度学习模型的可靠性,使用足够的测试数据对其一般行为和各种边界条件下的极端行为进行充分测试是十分必要的。在深度学
习模型测试研究中,如何生成具有代表性的,容易暴露模型潜在错误行为的测试数据已成为深度学习测试的一个研究重点。关于深度学习
模型测试集生成的研究工作可分为\textbf{测试数据生成和优选}两方面。


\begin{table}[htp]
	\renewcommand\arraystretch{1.5}
	\small
	\centering
	\caption{面向深度学习模型的测试数据生成方法}
	\label{tab:testingDataGen}
	% \begin{tabular}{cp{5cm}p{2cm}cp{2cm}}
	\begin{tabular}{cccccc}
		\toprule
		\textbf{序号} & \textbf{算法思想} & \textbf{评价方法}              & \textbf{测试数据} & \textbf{文献号}                                                             \\
		\midrule
		1             & 模糊测试          & 测试覆盖率、效率               & 图像、文本        & \cite{Odena2019TensorFuzz}\cite{Guo2018DLFuzz}\cite{xie2019coverage}        \\
		2             & 符号执行          & 测试覆盖率、像素重要性等       & 图像、代码        & \cite{Gopinath2018Symbolic}\cite{Sun2018Concolic}                           \\
		3             & 对抗攻击          & 准确率、失真度、人类对比评价等 & 图像、文本等      & \cite{Xiao2018Spatially}\cite{Wicker2018FeatureGuided}\cite{He2018Decision} \\
		\bottomrule
	\end{tabular}
\end{table}


在测试数据生成方面,现有方法可分为基于模糊测试、符号执行和对抗攻击这三类方法(见\cref{tab:testingDataGen})。其中,基于
模糊测试的方法通过随机或者特定规则将输入种子进行变异,观察模型在边界条件下是否会发生错误
\citess{Guo2018DLFuzz,Odena2019TensorFuzz,xie2019coverage,zhang2018deeproad}。Sun等人\citess{Sun2018Concolic}在测试覆盖
指标的基础上,结合具体执行和符号分析生成新的测试数据,以快速提高神经网络的测试覆盖率。Gopinath等人
\citess{Gopinath2018Symbolic}提出了一种轻量级的符号执行技术来测试图像分类模型,解决了重要像素的识别以及创建1像素和2像素
攻击等关键问题。基于对抗样本攻击的方法,很多研究通过向原始样本添加微小扰动的方式生成对抗样本,使深度学习模型做出错误预测
\citess{Xiao2018Spatially,He2018Decision,Wicker2018FeatureGuided}。

\begin{table}[htp]
	\renewcommand\arraystretch{1.5}
	\small
	\centering
	\caption{面向深度学习模型的测试数据优选方法}
	\label{tab:testingDataPri}
	\begin{tabular}{cccc}
		\toprule
		\textbf{序号} & \textbf{算法思想} & \textbf{测试对象} & \textbf{文献号}                                                                                                          \\
		\midrule
		1             & 变异测试方法      & 图像              & \cite{Wang2021Prioritizing}\cite{Ma2018DeepMutation}  \cite{Liu2022DeepState}                                            \\
		2             & 测试数据输出概率  & 图像、自然语言    & \cite{Byun2019Input}\cite{Shen2020MultipleBoundary}\cite{Feng2020DeepGini}\cite{Hu2022AnEmpirical}\cite{Gao2022Adaptive} \\
		\bottomrule
	\end{tabular}
\end{table}

在测试数据优选方面,由于深度学习模型的输入样本空间通常较大,而人工标注测试预言的成本较高,因此很难在系统部署前检测每个输
入样本的正确性。为了解决这个问题,部分研究工作从大规模无标注数据中选择一个测试子集来优先标注和测
试。\cref{tab:testingDataPri}总结了现有的测试数据选择方法,根据算法的思想可分为基于变异测试和模型输出概率的方法。其中,
变异测试通过设计针对深度学习模型的变异算子,从模型结构、参数、权重和输入等方面对模型进行变异,为每个测试数据生成多个模型
变异体,衡量测试数据对模型错误的检测能力,优选能够检测出较多变异体的测试数据来标注和测试
\citess{Ma2018DeepMutation,Wang2021Prioritizing,liu2019exploiting,Liu2022DeepState}。{基于模型的输出概率},部分研究工作
从置信度、不确定性、意外度、敏感度等角度估算测试数据的错误概率,优选可能导致模型预测错误的测试数据进行标注和测试
\citess{Feng2020DeepGini,Hu2022AnEmpirical}。

\textbf{在基于深度学习模型的安全攸关任务上,测试数据集生成方法DeepSuite\citess{xu2021deepsuite}对自动驾驶软件的转向
	(角)预测模型的测试进行了探索性尝试和研究(申请人作为该文章的第一作者,且文章已收录于IEEE T-ITS,中科院SCI一区期刊)}。该方
法主要基于多目标搜索对测试集种子进行多粒度变异和搜索,生成规模可控的具有高代表性的测试集,并在伯克利
DeepDrive(BDD)数据集和Udacity数据集上证明了该方法所生成的测试集具有高代表性和错误检测能力。\textbf{该研究工作说明申请
	人在该领域有坚实的研究基础},本项目将会继续沿着该学术方向在以下方面进行更为深入的研究:
\begin{itemize}
	\item \textbf{现有测试度量指标依赖于对网络结构和神经元值的覆盖,缺乏对模型决策行为和决策依据的理解}。与传统软件
	      测试不同,测试人员无法将测试度量指标和模型的待测功能联系起来,因此,虽然在现有度量指标的引导下能够生成导致模型预
	      测错误的测试数据,但测试人员无法根据测试结果反馈进行模型诊断和调试,导致测试方法与实际应用存在一定的距离。
	\item \textbf{现有测试方法存在面对大规模深度学习模型时伸缩性较差的共性问题,复杂的网络结构导致测试可解释性无法得
		      到更为深入的研究}。天津大学刘爽老师团队的综述论文\citess{survey}对深度神经网络测试研究进行了详细的调研和总
	      结,研究表明目前深度神经网络的测试过程需要较大的计算资源,如何提升对大规模深度神经网络的测试和理解,是一项
	      亟待解决的问题。
	\item \textbf{现有测试场景主要集中于白盒测试场景,而对受限场景下深度学习模型的黑盒测试研究较少,更缺乏针对此种情
		      况的可解释性研究}。现有测试方法假设测试人员能够掌握所有的训练数据和整个深度学习模型,但在许多场景
	      下,例如模型为商业公司私有或第三方机构提供时,测试者无法访问训练数据和模型内部结构,如何在黑盒测试
	      场景下对深度学习模型进行可解释测试,具有重要意义。
	\item \textbf{现有的测试集生成方法缺乏具有解释性的测试逻辑,难以提供能够被测试人员和用户所理解的整体检测报告}。面向深
	      度学习模型的系统性测试方法,不仅要评估模型在给定测试集上的性能表现,还需要建立测试结果与待测模型行为之间的
	      内在联系。现有的测试集生成方法缺乏能够被测试人员和用户所理解的测试逻辑,测试人员无法解释测试集的充分性和代
	      表性,用户很难凭借在给定测试集上的测试结果对模型产生足够的信任,针对深度学习模型的自动化可解释性
	      测试亟待研究。
\end{itemize}




% 因为写 demo,我把参考文献放这里了,真写本子的时候,还是要放在国内外概况那边
\begin{spacing}{1.3} % 行距
	\zihao{5} \songti
	\bibliographystyle{gbt7714-nsfc}
	\bibliography{ref,cai_refs}
	\vspace{11bp}
\end{spacing}
