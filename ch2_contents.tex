\subsection{研究内容}\label{ch2content}

本项目面向医疗卫生行业的数据分析需求,针对电子医疗记录分析存在的研究队列识别困
难、记录时间不规则、模型解释匮乏等问题,以深度学习为基础手段,研究电子医疗记录分
析建模的理论和方法,力争构建端到端的电子医疗记录分析方案,突破队列识别、EMR插补
和可解释性分析模型等关键技术,并在基于电子医疗记录的临床任务上验证本项目的研究成
果。

项目研究工作从队列识别、EMR插补、可解释分析模型和临床任务验证四个层次展开,本项
目的挑战、科学问题和研究内容关系如图~\ref{fig:ch2:rc}所示。各部分研究内容具体介
绍如下:

\begin{figure}[htp]
    \begin{small}
        \begin{center}
            \includegraphics[width=0.95\textwidth]{ch2_framework.pdf}
        \end{center}
        \caption{挑战、科学问题和研究内容关系图}
        \label{fig:ch2:rc}
    \end{small}
\end{figure}

\subsubsection{基于决策行为模拟的黑盒测试}

借鉴传统软件的路径覆盖测试方法,许多研究提出针对深度学习模型的覆盖性测试方法(参
见\ref{relatedwork} 国内外研究现状及发展动态分析),这些测试覆盖指标和方法主要是
针对白盒的场景,即测试者掌握所有的训练数据和整个深度学习模型,\textbf{但在许多场
景中,测试者无法访问训练数据和模型内部结构,但仍需要对模型的泛化能力进行测试,即
黑盒测试,如深度学习模型是某个公司私有的或者由第三方机构提供,他们只提供了接口或
者打包的可执行程序。}因此,本项目面向黑盒测试场景,研究基于决策行为模拟的黑盒测
试方法。

以分类模型为例,给定一个黑盒深度学习模型$\mathcal M$和测试数据集$\mathcal
D_{\text{test}}=\{x^{(i)},y^{(i)}\}$,测试者可以得到模型针对每个输入
$x^{(i)}$(input)的输出$\hat{y}^{(i)}$,通常是该输入属于各个类别的概率分布,除
此之外,测试者无法知道该模型的训练数据和内部结构。可见,深度学习模型黑盒测试面临
两个主要问题:
\begin{itemize}
    \item \textbf{如何建立模型有关的测试方法?}仅有的几个黑盒模型测试方法仅针对
    数据的多样性进行评估,无法准确反映模型的泛化能力,也不能给模型改进提供有效建
    议。
    \item \textbf{如何掌握模型的决策机制?}黑盒测试无法了解模型的结构,难以掌握模型的决策机制。
\end{itemize}

\begin{figure}[htp]
    \begin{small}
        \begin{center}
            \includegraphics[width=0.75\textwidth]{ch2_2Btest.pdf}
        \end{center}
        \caption{基于决策行为模拟的黑盒测试研究内容}
        \label{fig:ch2:2Btest}
    \end{small}
\end{figure}
本项目深度学习模型黑盒测试的主要研究内容如\cref{fig:ch2:2Btest}所示,\textbf{本项目首先
研究针对黑盒模型的预测行为模拟方法,拟利用知识萃取的方法,通过建立副本模型
$\mathcal M^\prime$(如决策树)萃取黑盒模型$\mathcal M$的知识,模拟$\mathcal M$
的预测行为;然后,在副本模型的基础上,本项目拟建立基于决策路径覆盖度的测试方法,
通过覆盖度指标反映模型的决策机制和泛化能力。}

\subsubsection{基于层次语义理解的白盒测试}

除了黑盒测试的场景,白盒测试的需求也非常多,如企业自己开发的深度学习模型,在白盒
测试中,测试者可以访问模型的内部结构$\mathcal M(\bm W, \bm b)$和训练数据集
$\mathcal D_{train}=\{(x^{(i)}, y^{(i)}\})$。目前,针对深度学习模型的白盒测试方
法可扩展性和可解释性较差,无法应用于大规模深度学习模型,如
BERT~\cite{kenton2019bert},MAE~\cite{he2021masked}等,而且测试结果无法给模型训
练提供有效反馈,辅助模型优化。因此,\textbf{本项目提出针对白盒深度学习模型的层次
语义理解方法,通过知识蒸馏技术将各层的决策语义融入``学生''模型中,保证``学生''模
型的决策路径与原白盒模型一致;然后围绕``学生''模型进行决策路径覆盖度测试,决策路
径具有较强的可解释性,可引导测试数据生成和模型优化。}

\begin{figure}[htp]
    \begin{small}
        \begin{center}
            \includegraphics[width=0.9\textwidth]{ch2_WBtest.pdf}
        \end{center}
        \caption{基于层次语义理解的白盒测试研究内容}
        \label{fig:ch2:WBtest}
    \end{small}
\end{figure}

本项目基于层次语义理解的白盒测试研究内容如\cref{fig:ch2:WBtest}所示,在白盒测试
的场景下,本项目首先研究如何从深度学习模型中抽取出各层(layer)的决策语义信息,
研究表明,深度学习模型对输入的决策是由粗粒度到细粒度的。\cref{fig:ch2:WBtest}举
例说明了从3组神经网络层的输出抽取各组网络的预测行为,例如:输入一个猫的图片,第1
组神经网络判断该图片是否是动物,接着,第2组神经网络判断该图片是否是猫科动物,最
后,第3组神经网络将该图片分类为猫。\textbf{本项目拟研究如何实现层次语义理解,自
动地从深度学习模型中抽取出各层的决策语义}。

在层次语义理解的基础上,\textbf{本项目拟利用知识蒸馏技术训练一个``学生''模型,通
过融入原白盒模型的决策语义,使知识蒸馏得到的``学生''的决策路径与原白盒模型一致,
且具有良好的可解释性}。此外,知识蒸馏得到的模型通常规模较小,和直接测试原白盒模
型相比,测试``学生''模型可有效提高测试效率。\textbf{在得到可解释的``学生''模型
后,本项目拟研究并提出针对该可解释模型的决策路径覆盖度,用于分析测试样本的多样性
和充分性}。


\subsubsection{基于反馈偏置的自适应测试集生成}
深度学习模型的测试依赖于有标签的测试集,以判断模型预测结果是否符合预期,然而,测
试集的标注成本通常较高。一方面,深度学习模型的测试输入标注通常依赖于人工经验,一
个测试数据需要多个经验丰富的标注人员来标出,以保证标注正确性;另一方面,为了测试
充分性和准确评估模型性能,测试集的规模通常较大,尽可能代表真实的测试数据分布,导
致标注成本较高。因此,{需要合理平衡测试集的规模和质量,在有限标注成本空间内选取
对样本空间具有高代表性的测试数据,生成规模相对较小的高质量测试集优先标注}。\textbf{本项
目拟提出一种可解释的自适应测试集生成方法,给定输入空间内大规模无标注测试数据,针
对充分性测试和模型性能评估两个测试目标,结合测试执行反馈,筛选出具有不同检测能力
的测试数据,生成指定规模的高质量测试集。}

\begin{figure}[htp]
    \begin{small}
        \begin{center}
            \includegraphics[width=0.95\textwidth]{ch2_TestSelection.pdf}
        \end{center}
        \caption{可解释预测模型研究内容}
        \label{fig:ch2:testselection}
    \end{small}
\end{figure}

如\cref{fig:ch2:testselection}所示,本项目拟在自适应测试中考虑可解释性,现有工作
缺乏对测试数据生成和测试结果反馈的可解释性,导致针对深度学习的有效测试信息较少,
难以辅助修复模型缺陷。结合前面研究内容,本项目可从黑盒和白盒两个角度将复杂神经网
络抽象为可解释蒸馏模型,对每个输入数据都输出具有可解释性的决策路径。针对给定大规
模无标注测试数据,可经过可解释蒸馏模型得到所有测试数据的决策路径分布,具有相同决
策路径的测试数据为一组。根据每组测试数据规模大小,采用聚类分析将具有相同决策路径
的测试数据分为多个类簇,利用基于最大平均差异法为从每簇测试数据中选取固定总数的测
试数据作为代表性数据,得到与无标注数据集有近似决策路径分布的初始子集。\textbf{本
项目拟从充分性测试和准确评估模型两个测试目标来制定自适应反馈机制,将复杂神经网络
抽象成可解释蒸馏模型,选取与大规模无标注测试数据具有近似决策路径分布的代表性数据
进行标注,为测试人员提供全阶段的可解释测试方法}。


\subsection{研究目标}\label{ch2target}

本项目从医疗卫生行业和人工智能结合的实际需求出发,针对队列识别泛化能力差、EMR插
补准确率低、以及预测模型可解释性匮乏等问题,研究队列识别、EMR插补和可解释预测模
型等问题,并将相关研究成果应用实际临床数据,检验本项目的研究成果在电子医疗记录预
测性分析中的应用效果,推动电子医疗记录分析的研究进展。

具体研究目标包括:

\textbf{在技术方面},本项目拟在以下四方面实现技术突破: (a) 提出基于表现型的队列
识别方法,充分利用已标注和未标注数据,提高队列识别模型的准确性和表示学习的泛化能
力;(b) 提出融合医学偏差的EMR自动插补方法,实现数据预处理,通过合理引入医学偏
差,提高EMR插补的准确性;(c) 提出可解释预测模型,在进行预测的同时,可以从模型中
得到与预测目标相关的特征和时间点;(d) 将本项目的研究成果应用于DKD患者病情进展预
测和重症监护室患者感染性休克预测。

\textbf{在成果形式方面},本项目力争在国内外高水平期刊、会议上发表论文8篇以上,全
部被SCI/EI检索,其中有重要影响的论文4篇以上;申请专利2项。

\textbf{在人才培养方面},通过本项目研究,培养深度学习、电子医疗记录分析、可解释人工智能等交叉领域的青年人才,拟培养研究生3-4人。

\subsection{拟解决的关键科学问题}

基于上述研究目标和研究内容,本项目拟在以下几个关键理论和技术问题上有所突破:

\subsubsection{在弱监督条件下构建泛化性强的患者层次表示,实现自动队列识别}

从海量、高维的电子医疗记录中识别研究队列是电子医疗记录分析的基础,由于电子医疗记
录的高维性和表现型的组合性,依靠医务人员人力筛选和标注所有符合预定表现型的队列数
据不具有可操作性。患者的表现型表示可以更准确地对应预定义表现型,提高队列识别准确
度,但仅基于少量队列标签的EMR数据集不足以让模型学习到好的数据分布,同时,传统半
监督方法在标注数据和未标注数据上分别训练,模型的训练更容易被未标注数据影响。因
此,如何合理利用大量未标注数据辅助建模,构建泛化性强的患者层次表示,实现电子医疗
记录的队列识别是本项目拟解决一个关键科学问题。

\subsubsection{建立医学偏差和缺失矩阵的对应关系,并将其融入电子医疗记录插补中}

电子医疗记录具有时间不规则性,直接在电子医疗记录上建模和训练难以得到较好的结果,
训练得到的模型通常容易过拟合,泛化性较差。在构建预测模型之前对电子医疗记录进行插
补具有重要意义,可以降低预测模型复杂度和训练难度。现有方法忽略了电子医疗记录中固
有的医学偏差,将其中的缺失值当作完全随机缺失对待,其插补结果不够准确。医学偏差并
不是数据质量问题或者噪声,反而对充分理解电子医疗记录非常重要。如何建立医学偏差和
记录缺失的对应关系,并将其融合到电子医疗记录插补中是本项目构建通用电子医疗记录处
理框架要解决的一个关键科学问题。

\subsubsection{融合特征重要性和时间关联性,建立电子医疗记录可解释分析模型}

模型的可解释性对医学领域的预测性分析至关重要,决定了模型是否能应用于临床实践。电
子医疗记录(动态特征)是一种特殊的多元时序数据,通常采用RNN模型建模,但在RNN学习
过程中,多个特征在不同时间点的取值都被糅合在RNN的隐含层中,无法总结得到特征对于
模型的意义,也无法计算出某条记录和预测结果的相关性。因此,如何融合特征重要性和时
间关联性,构建可解释预测模型,支持模型的全局解释和预测结果的回溯,也是本项目拟解
决的一个关键科学问题。